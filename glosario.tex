% ==============================================================================
% Acrónimos
% ==============================================================================

% A
\newacronym{api}{API}{Application Programming Interface}

% B
\newacronym{ble}{BLE}{Bluetooth Low Energy}

% C
\newacronym[longplural={Redes Neuronales Convolucionales}]{cnn}{CNN}{Red Neuronal Convolucional}

% E
\newacronym{esp}{ESP}{Espressif}

% F
\newacronym{fps}{FPS}{Frames Per Second}

% G
\newacronym{gps}{GPS}{Global Positioning System}

% I
\newacronym{ia}{IA}{Inteligencia Artificial}
\newacronym{iot}{IoT}{Internet de las Cosas}
\newacronym{ir}{IR}{Infrarrojo}

% M
\newacronym{ml}{ML}{Machine Learning}
\newacronym{mqtt}{MQTT}{Message Queuing Telemetry Transport}

% P
\newacronym{pir}{PIR}{Infrarrojo Pasivo}

% R
\newacronym[longplural={Redes Neuronales Convolucionales basadas en Regiones}]{rcnn}{R-CNN}{Red Neuronal Convolucional basada en Regiones}

% S
\newacronym{soc}{SoC}{System on Chip}

% Y
\newacronym{yolo}{YOLO}{You Only Look Once}

% ==============================================================================
% Glosario
% ==============================================================================

\newglossaryentry{AP-STA}{
	name=AP-STA,
	description={Modo de operación en el que un dispositivo actúa simultáneamente como Punto de Acceso (AP) y Estación (STA)}
}

\newglossaryentry{bounding-box}{
	name=bounding box,
	description={Rectángulo que delimita la ubicación de un objeto detectado en una imagen}
}

\newglossaryentry{camara-trampa}{
	name=cámara trampa,
	description={Dispositivo de captura de imágenes activado por movimiento, utilizado comúnmente para monitoreo de fauna silvestre}
}

\newglossaryentry{edge-computing}{
	name=edge computing,
	description={Paradigma de computación distribuida que procesa datos cerca de su fuente de origen, reduciendo latencia y ancho de banda}
}

\newglossaryentry{ESP-IDF}{
	name=ESP-IDF,
	description={Espressif IoT Development Framework, entorno oficial de desarrollo para microcontroladores ESP32}
}

\newglossaryentry{ESP-MDF}{
	name=ESP-MDF,
	description={Espressif Mesh Development Framework, framework para desarrollo de redes mesh sobre ESP-IDF}
}

\newglossaryentry{ESP-MESH}{
	name=ESP-MESH,
	description={Protocolo de red mesh desarrollado por Espressif para microcontroladores ESP32, basado en Wi-Fi}
}

\newglossaryentry{firmware}{
	name=firmware,
	description={Software de bajo nivel almacenado en la memoria de un dispositivo que controla su funcionamiento básico}
}

\newglossaryentry{inferencia}{
	name=inferencia,
	description={Proceso de utilizar un modelo de aprendizaje automático entrenado para realizar predicciones sobre nuevos datos}
}

\newglossaryentry{mesh}{
	name=mesh,
	description={Topología de red donde cada nodo puede conectarse con múltiples nodos vecinos, permitiendo rutas alternativas para la transmisión de datos}
}

\newglossaryentry{miniz}{
	name=miniz,
	description={Librería escrita en C de implementación ligera para compresión y descompresión de datos basada en el algoritmo zlib}
}

\newglossaryentry{nodo}{
	name=nodo,
	description={Dispositivo individual que forma parte de una red mesh}
}

\newglossaryentry{nodo-raiz}{
	name=nodo raíz,
	description={Nodo principal de una red mesh que actúa como puerta de enlace hacia redes externas}
}