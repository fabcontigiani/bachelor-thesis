% ==============================================================================
% Acrónimos
% ==============================================================================

% A
\newacronym{api}{API}{Application Programming Interface}

% C
\newacronym[longplural={Redes Neuronales Convolucionales}]{cnn}{CNN}{Red Neuronal Convolucional}

% F
\newacronym{fps}{FPS}{Frames Per Second}

% I
\newacronym{ia}{IA}{Inteligencia Artificial}
\newacronym{iot}{IoT}{Internet de las Cosas}
\newacronym{ir}{IR}{Infrarrojo}

% J
\newacronym{jpeg}{JPEG}{Joint Photographic Experts Group}

% L
\newacronym{lwip}{LwIP}{Lightweight IP}

% M
\newacronym{ml}{ML}{Machine Learning}

% P
\newacronym{pir}{PIR}{Infrarrojo Pasivo}

% S
\newacronym{soc}{SoC}{System on Chip}

% T
\newacronym{tcp}{TCP}{Transmission Control Protocol}

% Y
\newacronym{yolo}{YOLO}{You Only Look Once}

% ==============================================================================
% Glosario
% ==============================================================================

\newglossaryentry{AP-STA}{
	name=AP-STA,
	description={Modo de operación en el que un dispositivo actúa simultáneamente como Punto de Acceso (AP) y Estación (STA)}
}

\newglossaryentry{bounding-box}{
	name=bounding box,
	description={Rectángulo que delimita la ubicación de un objeto detectado en una imagen}
}

\newglossaryentry{camara-trampa}{
	name=cámara trampa,
	description={Dispositivo de captura de imágenes activado por movimiento, utilizado comúnmente para monitoreo de fauna silvestre}
}

\newglossaryentry{docker}{
	name=Docker,
	description={Plataforma de contenedorización que permite empaquetar aplicaciones junto con sus dependencias para facilitar el despliegue}
}

\newglossaryentry{docker-compose}{
	name=Docker Compose,
	description={Herramienta para definir y ejecutar aplicaciones Docker multi-contenedor mediante archivos YAML}
}

\newglossaryentry{django}{
	name=Django,
	description={Framework de desarrollo web de alto nivel escrito en Python, que sigue el patrón modelo-vista-plantilla}
}

\newglossaryentry{ESP-IDF}{
	name=ESP-IDF,
	description={Espressif IoT Development Framework, entorno oficial de desarrollo para microcontroladores ESP32}
}

\newglossaryentry{ESP-MDF}{
	name=ESP-MDF,
	description={Espressif Mesh Development Framework, framework para desarrollo de redes mesh sobre ESP-IDF}
}

\newglossaryentry{ESP-MESH}{
	name=ESP-MESH,
	description={Protocolo de red mesh desarrollado por Espressif para microcontroladores ESP32, basado en Wi-Fi}
}

\newglossaryentry{ESP32}{
	name=ESP32,
	description={Microcontrolador de bajo costo y bajo consumo con Wi-Fi y Bluetooth integrados, fabricado por Espressif Systems}
}

\newglossaryentry{firmware}{
	name=firmware,
	description={Software de bajo nivel almacenado en la memoria de un dispositivo que controla su funcionamiento básico}
}

\newglossaryentry{inferencia}{
	name=inferencia,
	description={Proceso de utilizar un modelo de aprendizaje automático entrenado para realizar predicciones sobre nuevos datos}
}

\newglossaryentry{litserve}{
	name=LitServe,
	description={Framework ligero para servir modelos de aprendizaje automático como APIs HTTP}
}

\newglossaryentry{mesh}{
	name=mesh,
	description={Topología de red donde cada nodo puede conectarse con múltiples nodos vecinos, permitiendo rutas alternativas para la transmisión de datos}
}

\newglossaryentry{miniz}{
	name=miniz,
	description={Librería escrita en C de implementación ligera para compresión y descompresión de datos basada en el algoritmo zlib}
}

\newglossaryentry{mwifi}{
	name=Mwifi,
	description={Componente del ESP-MDF que proporciona APIs de alto nivel para comunicación en redes ESP-MESH}
}

\newglossaryentry{nodo}{
	name=nodo,
	description={Dispositivo individual que forma parte de una red mesh}
}

\newglossaryentry{nodo-raiz}{
	name=nodo raíz,
	description={Nodo principal de una red mesh que actúa como puerta de enlace hacia redes externas}
}

\newglossaryentry{postgresql}{
	name=PostgreSQL,
	description={Sistema de gestión de bases de datos relacional de código abierto}
}

\newglossaryentry{speciesnet}{
	name=SpeciesNet,
	description={Modelo de detección y clasificación de fauna silvestre desarrollado por Google, basado en YOLOv5}
}

\newglossaryentry{telegram-bot}{
	name=bot de Telegram,
	description={Programa automatizado que interactúa con usuarios a través de la plataforma de mensajería Telegram}
}

\newglossaryentry{yolov5}{
	name=YOLOv5,
	description={Versión 5 del modelo You Only Look Once, arquitectura de red neuronal para detección de objetos en tiempo real}
}