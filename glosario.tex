% ==============================================================================
% Acrónimos
% ==============================================================================

% A
\newacronym{api}{API}{Application Programming Interface}
\newacronym{asa}{ASA}{Acrilonitrilo Estireno Acrilato}

% E
\newacronym{eap}{EAP}{Establecimiento Agropecuario Productivo}

% C
\newacronym[longplural={Redes Neuronales Convolucionales}]{cnn}{CNN}{Red Neuronal Convolucional}
\newacronym{cmos}{CMOS}{Complementary Metal-Oxide-Semiconductor}

% F
\newacronym{fdm}{FDM}{Modelado por Deposición Fundida}
\newacronym{fps}{FPS}{Frames Per Second}
\newacronym{freertos}{FreeRTOS}{Free Real-Time Operating System}

% I
\newacronym{ia}{IA}{Inteligencia Artificial}
\newacronym{iot}{IoT}{Internet de las Cosas}
\newacronym{ir}{IR}{Infrarrojo}

% J
\newacronym{jpeg}{JPEG}{Joint Photographic Experts Group}

% L
\newacronym{lwip}{LwIP}{Lightweight IP}

% M
\newacronym{ml}{ML}{Machine Learning}

% O
\newacronym{ota}{OTA}{Over-The-Air}

% P
\newacronym{pir}{PIR}{Infrarrojo Pasivo}
\newacronym{pla}{PLA}{Ácido Poliláctico}
\newacronym{petg}{PETG}{Tereftalato de Polietileno Glicol}

% R
\newacronym{rssi}{RSSI}{Received Signal Strength Indicator}

% S
\newacronym{soc}{SoC}{System on Chip}

% T
\newacronym{tcp}{TCP}{Transmission Control Protocol}

% U
\newacronym{udp}{UDP}{User Datagram Protocol}

% Y
\newacronym{yolo}{YOLO}{You Only Look Once}

% ==============================================================================
% Glosario
% ==============================================================================

\newglossaryentry{AP-STA}{
	name=AP-STA,
	description={Modo de operación en el que un dispositivo actúa simultáneamente como Punto de Acceso (AP) y Estación (STA)}
}

\newglossaryentry{bounding-box}{
	name=bounding box,
	description={Rectángulo que delimita la ubicación de un objeto detectado en una imagen}
}

\newglossaryentry{buck-converter}{
	name=convertidor buck,
	description={Circuito regulador de voltaje de tipo reductor (step-down) que convierte un voltaje de entrada mayor a un voltaje de salida menor con alta eficiencia. Utilizado para alimentar dispositivos electrónicos desde baterías de mayor voltaje}
}

\newglossaryentry{18650}{
	name=batería 18650,
	plural=baterías 18650,
	description={Celda de batería de iones de litio con formato cilíndrico de \qty{18}{mm} de diámetro y \qty{65}{mm} de longitud. Ampliamente utilizada en dispositivos electrónicos portátiles por su alta densidad energética y capacidad de recarga}
}

\newglossaryentry{bosque-atlantico}{
	name=Bosque Atlántico,
	description={Ecorregión de bosques subtropicales húmedos que se extiende por Brasil, Paraguay y Argentina, conocida también como Mata Atlántica, uno de los biomas más biodiversos del planeta}
}

\newglossaryentry{corredor-verde}{
	name=Corredor Verde,
	description={Corredor biológico de aproximadamente \num{1.1} millones de hectáreas en la provincia de Misiones, Argentina, que conecta áreas protegidas para permitir el desplazamiento de fauna silvestre y mantener la conectividad genética de las poblaciones}
}

\newglossaryentry{hotspot-biodiversidad}{
	name=hotspot de biodiversidad,
	description={Región biogeográfica que alberga una concentración excepcional de especies endémicas y que ha experimentado una pérdida significativa de hábitat, concepto introducido por Norman Myers en 1988}
}

\newglossaryentry{monumento-natural}{
	name=Monumento Natural,
	description={Categoría de protección legal en Argentina que confiere a una especie o área natural el máximo nivel de protección, prohibiendo cualquier forma de caza, captura o comercio}
}

\newglossaryentry{mata-atlantica}{
	name=Mata Atlántica,
	description={Nombre en portugués del Bosque Atlántico, utilizado principalmente en Brasil donde se encuentra el \qty{92}{\percent} de la extensión original de este bioma}
}

\newglossaryentry{servicios-ecosistemicos}{
	name=servicios ecosistémicos,
	description={Beneficios que los ecosistemas proporcionan a la sociedad humana, incluyendo regulación climática, provisión de agua limpia, control de erosión, polinización y secuestro de carbono}
}

\newglossaryentry{fragmentacion-habitat}{
	name=fragmentación del hábitat,
	description={Proceso por el cual un ecosistema continuo se divide en parches aislados, reduciendo la conectividad entre poblaciones de especies y afectando su viabilidad genética y capacidad de dispersión}
}

\newglossaryentry{trafico-fauna}{
	name=tráfico de fauna,
	description={Comercio ilegal de animales silvestres vivos o sus partes, considerado la cuarta actividad ilícita más lucrativa a nivel mundial y una de las principales causas de pérdida de biodiversidad}
}

\newglossaryentry{selva-paranaense}{
	name=selva paranaense,
	description={Nombre alternativo para la Selva Misionera, que forma parte del bioma del Bosque Atlántico y se extiende por la cuenca del río Paraná en Argentina, Paraguay y Brasil}
}

\newglossaryentry{reserva-biosfera}{
	name=reserva de biosfera,
	description={Área protegida reconocida por la UNESCO bajo el programa El Hombre y la Biosfera (MAB), que integra conservación de la biodiversidad con uso sostenible de recursos naturales}
}

\newglossaryentry{alerta-temprana}{
	name=alerta temprana,
	description={Sistema de detección y notificación que permite identificar y comunicar eventos críticos con mínima latencia, posibilitando respuestas preventivas antes de que el daño se materialice}
}

\newglossaryentry{grado-proteccion-ip}{
	name=grado de protección IP,
	description={Código estándar definido por la norma IEC 60529 que indica el nivel de protección de un dispositivo contra la intrusión de sólidos (primer dígito) y líquidos (segundo dígito). Por ejemplo, IP67 indica protección total contra polvo y contra inmersión temporal en agua}
}

\newglossaryentry{deep-sleep}{
	name=deep sleep,
	description={Modo de bajo consumo extremo en microcontroladores donde la mayoría de los periféricos y el procesador principal se apagan, manteniendo solo un núcleo mínimo capaz de despertar el sistema ante eventos externos como señales de sensores o temporizadores}
}

\newglossaryentry{power-save}{
	name=power save,
	description={Modo de ahorro de energía en redes ESP-MESH donde los nodos reducen su consumo limitando el tiempo de transmisión Wi-Fi activa, manteniendo el CPU funcionando para procesar eventos}
}

\newglossaryentry{autocuracion}{
	name=autocuración,
	description={Capacidad de una red mesh para reconfigurarse automáticamente cuando un nodo falla, permitiendo que los nodos afectados encuentren rutas alternativas sin intervención manual}
}

\newglossaryentry{callback}{
	name=callback,
	description={Función que se pasa como argumento a otra función y que será invocada cuando ocurra un evento específico, permitiendo la programación asíncrona y orientada a eventos}
}

\newglossaryentry{kanban}{
	name=Kanban,
	description={Metodología ágil de gestión de proyectos que visualiza el flujo de trabajo mediante un tablero con columnas, limita el trabajo en progreso y promueve la entrega continua de valor}
}

\newglossaryentry{despliegue-continuo}{
	name=despliegue continuo,
	description={Práctica de ingeniería de software que automatiza el proceso de publicación de cambios de código a producción, permitiendo entregas frecuentes y confiables. También conocido como Continuous Deployment (CD)}
}

\newglossaryentry{git}{
	name=Git,
	description={Sistema de control de versiones distribuido de código abierto, diseñado para gestionar proyectos de software de cualquier tamaño con velocidad y eficiencia}
}

\newglossaryentry{github-actions}{
	name=GitHub Actions,
	description={Plataforma de integración y despliegue continuo (CI/CD) integrada en GitHub que permite automatizar flujos de trabajo de desarrollo, pruebas y despliegue}
}

\newglossaryentry{gcode}{
	name=código G,
	description={Lenguaje de programación utilizado para controlar máquinas de fabricación por control numérico, incluyendo impresoras 3D. Define la trayectoria de la herramienta, velocidades y temperaturas}
}

\newglossaryentry{aprendizaje-automatico}{
	name=aprendizaje automático,
	description={Rama de la inteligencia artificial que permite a los sistemas aprender y mejorar automáticamente a partir de datos sin ser programados explícitamente, también conocido como machine learning}
}

\newglossaryentry{camara-trampa}{
	name=cámara trampa,
	description={Dispositivo de captura de imágenes activado por movimiento, utilizado comúnmente para monitoreo de fauna silvestre}
}

\newglossaryentry{carcasa}{
	name=carcasa,
	description={Estructura protectora que alberga los componentes electrónicos de un dispositivo, protegiéndolos del ambiente externo}
}

\newglossaryentry{cad}{
	name=CAD,
	description={Diseño Asistido por Computadora (Computer-Aided Design), software utilizado para crear modelos 2D y 3D de objetos}
}

\newglossaryentry{caza-furtiva}{
	name=caza furtiva,
	description={Práctica ilegal de captura o matanza de animales silvestres sin autorización, a menudo de especies protegidas o en peligro de extinción}
}

\newglossaryentry{clasificacion-taxonomica}{
	name=clasificación taxonómica,
	description={Proceso de identificar y categorizar organismos biológicos en grupos jerárquicos como familia, género y especie}
}

\newglossaryentry{codigo-abierto}{
	name=código abierto,
	description={Software cuyo código fuente está disponible públicamente para que cualquiera pueda estudiarlo, modificarlo y distribuirlo}
}

\newglossaryentry{deteccion-objetos}{
	name=detección de objetos,
	description={Técnica de visión por computadora que identifica y localiza objetos dentro de una imagen, generalmente mediante bounding boxes}
}

\newglossaryentry{tiempo-real}{
	name=tiempo real,
	description={Procesamiento de datos que ocurre con latencia mínima, permitiendo respuestas inmediatas a eventos}
}

\newglossaryentry{docker}{
	name=Docker,
	description={Plataforma de contenedorización que permite empaquetar aplicaciones junto con sus dependencias para facilitar el despliegue}
}

\newglossaryentry{docker-compose}{
	name=Docker Compose,
	description={Herramienta para definir y ejecutar aplicaciones Docker multi-contenedor mediante archivos YAML}
}

\newglossaryentry{django}{
	name=Django,
	description={Framework de desarrollo web de alto nivel escrito en Python, que sigue el patrón modelo-vista-plantilla}
}

\newglossaryentry{edge-computing}{
	name=edge computing,
	description={Paradigma de computación distribuida que acerca el procesamiento y el almacenamiento de datos a la ubicación donde se necesitan, con el fin de mejorar los tiempos de respuesta y ahorrar ancho de banda, también conocido como computación en el borde}
}

\newglossaryentry{ESP-IDF}{
	name=ESP-IDF,
	description={Espressif IoT Development Framework, entorno oficial de desarrollo para microcontroladores ESP32}
}

\newglossaryentry{ESP-MDF}{
	name=ESP-MDF,
	description={Espressif Mesh Development Framework, framework para desarrollo de redes mesh sobre ESP-IDF}
}

\newglossaryentry{ESP-MESH}{
	name=ESP-MESH,
	description={Protocolo de red mesh desarrollado por Espressif para microcontroladores ESP32, basado en Wi-Fi}
}

\newglossaryentry{ESP32}{
	name=ESP32,
	description={Microcontrolador de bajo costo y bajo consumo con Wi-Fi y Bluetooth integrados, fabricado por Espressif Systems}
}

\newglossaryentry{firmware}{
	name=firmware,
	description={Software de bajo nivel almacenado en la memoria de un dispositivo que controla su funcionamiento básico}
}

\newglossaryentry{inferencia}{
	name=inferencia,
	description={Proceso de utilizar un modelo de aprendizaje automático entrenado para realizar predicciones sobre nuevos datos}
}

\newglossaryentry{impresion-3d}{
	name=impresión 3D,
	description={Proceso de fabricación aditiva que crea objetos tridimensionales mediante la deposición de material capa por capa}
}

\newglossaryentry{litserve}{
	name=LitServe,
	description={Framework ligero para servir modelos de aprendizaje automático como APIs HTTP}
}

\newglossaryentry{lora}{
	name=LoRa,
	description={Tecnología de comunicación inalámbrica de largo alcance (Long Range) y bajo consumo de energía, que utiliza modulación por espectro ensanchado}
}

\newglossaryentry{lorawan}{
	name=LoRaWAN,
	description={Protocolo de red que utiliza la tecnología LoRa para conectar dispositivos a internet, optimizado para bajo consumo y comunicaciones de largo alcance}
}

\newglossaryentry{mesh}{
	name=mesh,
	description={Topología de red donde cada nodo puede conectarse con múltiples nodos vecinos, permitiendo rutas alternativas para la transmisión de datos}
}

\newglossaryentry{megadetector}{
	name=MegaDetector,
	description={Modelo de detección de objetos desarrollado por Microsoft para identificar automáticamente animales, personas y vehículos en imágenes de cámaras trampa}
}

\newglossaryentry{miniz}{
	name=miniz,
	description={Librería escrita en C de implementación ligera para compresión y descompresión de datos basada en el algoritmo zlib}
}

\newglossaryentry{mwifi}{
	name=Mwifi,
	description={Componente del ESP-MDF que proporciona APIs de alto nivel para comunicación en redes ESP-MESH}
}

\newglossaryentry{nodo}{
	name=nodo,
	description={Dispositivo individual que forma parte de una red mesh}
}

\newglossaryentry{nodo-raiz}{
	name=nodo raíz,
	description={Nodo principal de una red mesh que actúa como puerta de enlace hacia redes externas}
}

\newglossaryentry{postgresql}{
	name=PostgreSQL,
	description={Sistema de gestión de bases de datos relacional de código abierto}
}

\newglossaryentry{pytorch}{
	name=PyTorch,
	description={Biblioteca de aprendizaje automático de código abierto basada en la biblioteca Torch, utilizada para aplicaciones como visión computacional y procesamiento de lenguaje natural}
}

\newglossaryentry{speciesnet}{
	name=SpeciesNet,
	description={Modelo de detección y clasificación de fauna silvestre desarrollado por Google, basado en YOLOv5}
}

\newglossaryentry{sensor-pir}{
	name=sensor PIR,
	description={Sensor de Infrarrojo Pasivo que detecta cambios en la radiación infrarroja emitida por objetos en movimiento, comúnmente utilizado para detectar presencia de animales o personas}
}

\newglossaryentry{selva-misionera}{
	name=Selva Misionera,
	description={Porción argentina del Bosque Atlántico, ubicada principalmente en la Provincia de Misiones, caracterizada por su alta biodiversidad y vegetación subtropical}
}

\newglossaryentry{telegram-bot}{
	name=bot de Telegram,
	description={Programa automatizado que interactúa con usuarios a través de la plataforma de mensajería Telegram}
}

\newglossaryentry{yolov5}{
	name=YOLOv5,
	description={Versión 5 del modelo You Only Look Once, arquitectura de red neuronal para detección de objetos en tiempo real}
}

\newglossaryentry{vision-computadora}{
	name=visión por computadora,
	description={Campo de la inteligencia artificial que permite a las computadoras interpretar y comprender información visual del mundo real}
}