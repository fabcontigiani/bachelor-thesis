% Tipo de documento y márgenes
\documentclass[a4paper, 12pt]{report}
\usepackage[top=3cm, bottom=3cm, left = 2cm, right = 2cm]{geometry}

% Codificación y lenguaje
\usepackage[utf8]{inputenc}
\usepackage{csquotes}
\usepackage[spanish]{babel}

% Imágenes
\usepackage{graphicx}
\usepackage{subcaption}

% Bibliografía
\usepackage[nottoc]{tocbibind}
\usepackage[style=ieee, backend=biber]{biblatex}
\addbibresource{referencias.bib}

% Links
\usepackage{hyperref}
\hypersetup{colorlinks=true,linkcolor=black,citecolor=black,urlcolor=blue}

% Glosario
\usepackage{datatool}[=v2.32]
\usepackage[toc,acronym]{glossaries}
\setacronymstyle{long-short}
\makenoidxglossaries
\loadglsentries{glosario}

\usepackage[locale=DE]{siunitx}

\title{Sistema de Monitoreo y Alerta Temprana basado en Inteligencia Artificial
para Áreas Protegidas}
\author{Autores:\\Fabrizio Martin Contigiani\\Gabriel Orlando Da
Silva Schmies \\\\Tutor:\\Dr. Ing. Sergio Eduardo Moya}
\date{\today}

\begin{document}

\maketitle

\begin{abstract}
	El presente trabajo describe el diseño e implementación de un sistema
	de monitoreo y alerta temprana para áreas protegidas, que combina
	tecnologías de \gls{iot} con \gls{ia}
	para la detección automática de fauna silvestre, personas y vehículos.

	El sistema está compuesto por una red de \glspl{nodo} de captura basados en
	microcontroladores \gls{ESP32} equipados con cámaras, los cuales se comunican
	mediante el protocolo \gls{ESP-MESH} para transmitir imágenes hacia un
	\gls{nodo-raiz}. Este nodo actúa como puerta de enlace, reenviando las imágenes
	a un servidor remoto a través de \gls{tcp}/IP.

	En el servidor, las imágenes son procesadas por un servicio de
	\gls{inferencia} basado en \gls{speciesnet}, un modelo de detección de objetos
	desarrollado por Google que utiliza \gls{yolov5}. Este modelo permite
	identificar y clasificar especies de fauna silvestre, así como detectar
	la presencia de humanos y vehículos, generando alertas automáticas
	ante posibles intrusiones.

	La arquitectura del servidor incluye una aplicación web desarrollada
	en \gls{django} para la gestión de imágenes, un \gls{telegram-bot} para el
	envío de notificaciones en tiempo real, y una base de datos \gls{postgresql}
	para el almacenamiento persistente. Todo el sistema está
	contenedorizado mediante \gls{docker} para facilitar su despliegue.

	Los resultados demuestran la viabilidad de implementar un sistema de
	vigilancia inteligente de bajo costo para áreas protegidas, capaz de
	operar de manera autónoma y alertar a los administradores ante
	eventos relevantes.

	\vspace{2em}
	\noindent\textbf{Palabras Clave} - Cámaras Trampa, Internet de las Cosas,
	Inteligencia Artificial, Monitoreo de Fauna, Detección de Intrusos,
	ESP-MESH, SpeciesNet, YOLO
\end{abstract}

\tableofcontents

\listoffigures

\listoftables

\printnoidxglossary[type=main]
\printnoidxglossary[type=\acronymtype]

% ==============================================================================
% Capítulo 1: Introducción
% ==============================================================================
\chapter{Introducción}

La conservación de la fauna silvestre y la protección de áreas naturales
representan desafíos críticos en la actualidad. La pérdida de biodiversidad,
la \gls{caza-furtiva} y la intrusión humana en ecosistemas protegidos amenazan
el equilibrio ecológico y la supervivencia de numerosas especies. Ante
esta problemática, surge la necesidad de implementar sistemas de monitoreo
que permitan vigilar estas áreas de manera continua y eficiente.

Tradicionalmente, el monitoreo de fauna silvestre se ha realizado mediante
\glspl{camara-trampa}, dispositivos que capturan imágenes cuando detectan
movimiento. Sin embargo, estos sistemas convencionales presentan limitaciones
significativas: requieren revisión manual periódica, no permiten alertas
en \gls{tiempo-real}, y generan grandes volúmenes de datos que deben ser
analizados manualmente por expertos.

El avance de tecnologías como el \gls{iot} y la \gls{ia} ofrece nuevas
posibilidades para superar estas limitaciones. La combinación de redes
de sensores inalámbricos con algoritmos de \gls{deteccion-objetos} permite
desarrollar sistemas capaces de identificar automáticamente fauna silvestre,
personas y vehículos, generando alertas inmediatas ante eventos relevantes.

El presente trabajo propone el diseño e implementación de un sistema de
monitoreo y alerta temprana para áreas protegidas, que integra una red
de nodos de captura basados en \gls{ESP32} comunicados mediante \gls{ESP-MESH},
un servidor de procesamiento con \gls{ia} basado en \gls{speciesnet}, y
un sistema de notificaciones a través de \gls{telegram-bot}.

\section{Contexto y motivación}

Las áreas protegidas enfrentan amenazas constantes que van desde la caza
ilegal de especies en peligro hasta la intrusión de personas no autorizadas
y vehículos en zonas restringidas. Los guardaparques y administradores
de estas áreas frecuentemente carecen de los recursos humanos y tecnológicos
necesarios para mantener una vigilancia efectiva sobre extensas superficies
de terreno, muchas veces en ubicaciones remotas con acceso limitado a
infraestructura de comunicaciones.

Las cámaras trampa tradicionales, aunque útiles para la investigación
científica, no fueron diseñadas para la vigilancia en tiempo real. Las
imágenes capturadas permanecen almacenadas en tarjetas de memoria que
deben ser recolectadas físicamente, lo que implica visitas frecuentes
al campo y retrasos significativos entre la captura de un evento y su
descubrimiento. Para cuando se detecta una intrusión o un acto de \gls{caza-furtiva},
los responsables ya se encuentran lejos del área.

Es cierto que en la actualidad existen cámaras trampa más modernas que
incorporan conectividad celular o satelital, permitiendo el envío remoto
de imágenes. Sin embargo, estos dispositivos suelen tener un costo
significativamente mayor, lo que limita su adopción masiva especialmente
en países en vías de desarrollo, donde paradójicamente se concentra gran
parte de la biodiversidad mundial. Además, muchas de estas soluciones
comerciales dependen de servicios en la nube propietarios con costos
de suscripción recurrentes.

La motivación de este proyecto surge de la necesidad de transformar el
paradigma del monitoreo pasivo hacia un sistema activo de vigilancia
inteligente, pero de manera accesible y de bajo costo. Un sistema que
no solo capture imágenes, sino que las transmita en \gls{tiempo-real}, las
analice automáticamente mediante algoritmos de \gls{ia}, y genere alertas
inmediatas cuando se detecten eventos de interés, ya sea la presencia
de fauna silvestre para fines de investigación, o la detección de intrusos
para fines de seguridad. Todo esto utilizando componentes de hardware
económicos y software de \gls{codigo-abierto}.

\section{Estructura del documento}

El presente documento se organiza en diez capítulos que describen de
manera progresiva el desarrollo del sistema propuesto:

\begin{description}
	\item[Capítulo 1 - Introducción:] Presenta el contexto general del
	      proyecto, la motivación y la estructura del documento.

	\item[Capítulo 2 - Antecedentes:] Revisa trabajos relacionados,
	      soluciones comerciales existentes y el estado del arte en sistemas
	      de monitoreo de fauna y detección de intrusos.

	\item[Capítulo 3 - Planteamiento del Problema:] Describe la
	      problemática de las áreas protegidas, las limitaciones de los
	      sistemas tradicionales y la justificación del proyecto.

	\item[Capítulo 4 - Objetivos y Alcance:] Define los objetivos
	      generales y específicos, así como el alcance y las limitaciones
	      del trabajo.

	\item[Capítulo 5 - Marco Teórico:] Presenta los fundamentos teóricos
	      sobre \gls{iot}, redes \gls{mesh}, \gls{ia} aplicada a visión por
	      computadora, y las tecnologías utilizadas.

	\item[Capítulo 6 - Metodología:] Describe el enfoque metodológico,
	      las etapas de desarrollo y las herramientas empleadas.

	\item[Capítulo 7 - Diseño del Sistema:] Detalla la arquitectura
	      general, el diseño del hardware, la red \gls{mesh}, el servicio de
	      detección y el servidor de aplicación.

	\item[Capítulo 8 - Implementación:] Presenta la implementación de
	      cada componente: nodos \gls{mesh}, nodo raíz, servicio de detección
	      y servidor de aplicación.

	\item[Capítulo 9 - Pruebas y Resultados:] Documenta las pruebas
	      realizadas y analiza los resultados obtenidos en conectividad,
	      detección, rendimiento y consumo energético.

	\item[Capítulo 10 - Conclusiones:] Resume las conclusiones generales,
	      los aportes del trabajo, trabajos futuros y recomendaciones.
\end{description}

% ==============================================================================
% Capítulo 2: Antecedentes
% ==============================================================================
\chapter{Antecedentes}

\section{Trabajos relacionados}

\section{Soluciones comerciales existentes}

\section{Estado del arte}

\section{Análisis comparativo}

% ==============================================================================
% Capítulo 3: Planteamiento del Problema
% ==============================================================================
\chapter{Planteamiento del Problema}

\section{Áreas protegidas y conservación de fauna silvestre}

\section{Problemática de la vigilancia en áreas remotas}

\section{Sistemas de monitoreo tradicionales}
\subsection{Cámaras trampa convencionales}
\subsection{Limitaciones actuales}

\section{Necesidad de detección de intrusos}

\section{Justificación del proyecto}

% ==============================================================================
% Capítulo 4: Objetivos y Alcance
% ==============================================================================
\chapter{Objetivos y Alcance}

\section{Objetivo general}

\section{Objetivos específicos}

\section{Alcance del proyecto}

\section{Limitaciones}

% ==============================================================================
% Capítulo 5: Marco Teórico
% ==============================================================================
\chapter{Marco Teórico}

\section{Internet de las Cosas (IoT)}
\subsection{Arquitecturas IoT}
\subsection{Protocolos de comunicación inalámbrica}

\section{Redes Mesh}
\subsection{Topologías de red}
\subsection{ESP-MESH y Mwifi}

\section{Inteligencia Artificial aplicada a visión por computadora}
\subsection{Redes neuronales convolucionales (CNN)}
\subsection{Detección de objetos con YOLO}
\subsection{SpeciesNet de Google}

\section{Tecnologías de desarrollo}
\subsection{Microcontroladores ESP32}
\subsection{ESP-IDF y ESP-MDF}
\subsection{Contenedorización con Docker}
\subsection{Framework Django}

% ==============================================================================
% Capítulo 6: Metodología
% ==============================================================================
\chapter{Metodología}

\section{Enfoque metodológico}

\section{Etapas del desarrollo}

\section{Herramientas y tecnologías utilizadas}

\section{Métricas de evaluación}

% ==============================================================================
% Capítulo 7: Diseño del Sistema
% ==============================================================================
\chapter{Diseño del Sistema}

\section{Arquitectura general}

\section{Diseño del hardware}
\subsection{Selección de componentes}
\subsection{Nodo de captura con cámara}
\subsection{Nodo raíz}
\subsection{Alimentación y consumo energético}

\section{Diseño de la red mesh}
\subsection{Topología de la red}
\subsection{Protocolo de comunicación}
\subsection{Formato de datos}

\section{Diseño del servicio de detección}
\subsection{Servidor de inferencia con SpeciesNet}
\subsection{Detección de animales, humanos y vehículos}
\subsection{Anotación de imágenes con bounding boxes}

\section{Diseño del servidor de aplicación}
\subsection{Arquitectura de servicios}
\subsection{Gestión de imágenes}
\subsection{Interfaz web}
\subsection{Bot de Telegram y sistema de alertas}

% ==============================================================================
% Capítulo 8: Implementación
% ==============================================================================
\chapter{Implementación}

\section{Nodo mesh (mesh-node)}
\subsection{Firmware del nodo de captura}
\subsection{Captura de imágenes}
\subsection{Compresión y transmisión}

\section{Nodo raíz (root-node)}
\subsection{Firmware del nodo raíz}
\subsection{Conexión con servidor TCP}
\subsection{Gestión de la red mesh}

\section{Servicio de detección (wildlife-detection)}
\subsection{Contenedor Docker con SpeciesNet}
\subsection{API de inferencia con LitServe}
\subsection{Procesamiento de imágenes}

\section{Servidor de aplicación (server)}
\subsection{Aplicación Django}
\subsection{Integración con SpeciesNet}
\subsection{Bot de Telegram y sistema de alertas}
\subsection{Base de datos PostgreSQL}
\subsection{Despliegue con Docker Compose}

% ==============================================================================
% Capítulo 9: Pruebas y Resultados
% ==============================================================================
\chapter{Pruebas y Resultados}

\section{Ambiente de pruebas}

\section{Pruebas de conectividad y red mesh}
\subsection{Alcance de la red}
\subsection{Latencia de transmisión}
\subsection{Estabilidad de la conexión}

\section{Pruebas de detección}
\subsection{Detección de fauna silvestre}
\subsection{Detección de humanos}
\subsection{Detección de vehículos}

\section{Evaluación del modelo de IA}
\subsection{Precisión y recall}
\subsection{Tiempo de inferencia}

\section{Pruebas de consumo energético}

\section{Pruebas del sistema de alertas}
\subsection{Tiempo de respuesta}

\section{Análisis de resultados}

% ==============================================================================
% Capítulo 10: Conclusiones
% ==============================================================================
\chapter{Conclusiones}

\section{Conclusiones generales}

\section{Aportes del trabajo}

\section{Trabajos futuros}

\section{Recomendaciones}

% ==============================================================================
% Referencias
% ==============================================================================
\printbibliography[heading=bibintoc]

% ==============================================================================
% Anexos
% ==============================================================================
\appendix

\chapter{Esquemáticos del hardware}

\chapter{Código fuente relevante}
\section{Firmware del nodo mesh}
\section{Firmware del nodo raíz}
\section{Servidor de detección}
\section{Aplicación Django}

\chapter{Manual de instalación y configuración}
\section{Configuración del firmware}
\section{Despliegue del servidor}
\section{Configuración del bot de Telegram}

\chapter{Manual de usuario}

\chapter{Especificaciones técnicas}

\chapter{Análisis de viabilidad económica}

\end{document}