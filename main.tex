% Tipo de documento y márgenes
\documentclass[a4paper, 12pt]{report}
\usepackage[top=3cm, bottom=3cm, left = 2cm, right = 2cm]{geometry}

% Codificación y lenguaje
\usepackage[utf8]{inputenc}
\usepackage{csquotes}
\usepackage[spanish,es-tabla]{babel}

% Imágenes
\usepackage{graphicx}
\usepackage[labelfont=bf]{caption}
\usepackage{subcaption}

% Tablas
\usepackage{tabularray}

% Bibliografía
\usepackage[nottoc]{tocbibind}
\usepackage[style=ieee, backend=biber]{biblatex}
\addbibresource{referencias.bib}

% Links
\usepackage{hyperref}
\hypersetup{colorlinks=true,linkcolor=black,citecolor=black,urlcolor=blue}

% Glosario
\usepackage{datatool}[=v2.32]
\usepackage[toc,acronym]{glossaries}
\setacronymstyle{long-short}
\makenoidxglossaries
\loadglsentries{glosario}

\usepackage[locale=DE]{siunitx}

\title{Sistema de Monitoreo y Alerta Temprana basado en Inteligencia Artificial
para Áreas Protegidas}
\author{Autores:\\Fabrizio Martin Contigiani\\Gabriel Orlando Da
Silva Schmies \\\\Tutor:\\Dr. Ing. Sergio Eduardo Moya}
\date{\today}

\begin{document}

\maketitle

\begin{abstract}
	\hspace{1.5em}El presente trabajo describe el diseño e implementación de un sistema
	de monitoreo y alerta temprana para áreas protegidas de la
	\gls{selva-misionera}, que combina tecnologías de \gls{iot} con \gls{ia}
	para la detección automática de fauna silvestre, personas y vehículos.

	El sistema está compuesto por una red de \glspl{nodo} de captura basados en
	microcontroladores \gls{ESP32} equipados con cámaras, los cuales se comunican
	mediante el protocolo \gls{ESP-MESH} para transmitir imágenes hacia un
	\gls{nodo-raiz}. Este nodo actúa como puerta de enlace, reenviando las imágenes
	a un servidor remoto a través de \gls{tcp}/IP.

	En el servidor, las imágenes son procesadas por un servicio de
	\gls{inferencia} basado en \gls{speciesnet}, un modelo de detección de objetos
	desarrollado por Google que utiliza \gls{yolov5}. Este modelo permite
	identificar y clasificar especies de fauna silvestre, así como detectar
	la presencia de humanos y vehículos, generando alertas automáticas
	ante posibles intrusiones.

	La arquitectura del servidor incluye una aplicación web desarrollada
	en \gls{django} para la gestión de imágenes, un \gls{telegram-bot} para el
	envío de notificaciones en tiempo real, y una base de datos \gls{postgresql}
	para el almacenamiento persistente. Todo el sistema está
	contenedorizado mediante \gls{docker} para facilitar su despliegue.

	Los resultados demuestran la viabilidad de implementar un sistema de
	vigilancia inteligente de bajo costo para áreas protegidas, capaz de
	operar de manera autónoma y alertar a los administradores ante
	eventos relevantes.

	\vspace{2em}
	\noindent\textbf{Palabras Clave} - Cámaras Trampa, Internet de las Cosas,
	Inteligencia Artificial, Monitoreo de Fauna, Detección de Intrusos,
	ESP-MESH, SpeciesNet, Selva Misionera
\end{abstract}

\tableofcontents

\listoffigures

\listoftables

\printnoidxglossary[type=main]
\printnoidxglossary[type=\acronymtype]

% ==============================================================================
% Capítulo 1: Introducción
% ==============================================================================
\chapter{Introducción}

La conservación de la fauna silvestre y la protección de áreas naturales
representan desafíos críticos en la actualidad. En regiones de alta
biodiversidad como la \gls{selva-misionera}, la pérdida de hábitat,
la \gls{caza-furtiva} y la intrusión humana en ecosistemas protegidos amenazan
el equilibrio ecológico y la supervivencia de numerosas especies. Ante
esta problemática, surge la necesidad de implementar sistemas de monitoreo
que permitan vigilar estas áreas de manera continua y eficiente.

Tradicionalmente, el monitoreo de fauna silvestre se ha realizado mediante
\glspl{camara-trampa}, dispositivos que capturan imágenes cuando detectan
movimiento. Sin embargo, estos sistemas convencionales presentan limitaciones
significativas: requieren revisión manual periódica, no permiten alertas
en \gls{tiempo-real}, y generan grandes volúmenes de datos que deben ser
analizados manualmente por expertos.

El avance de tecnologías como el \gls{iot} y la \gls{ia} ofrece nuevas
posibilidades para superar estas limitaciones. La combinación de redes
de sensores inalámbricos con algoritmos de \gls{deteccion-objetos} permite
desarrollar sistemas capaces de identificar automáticamente fauna silvestre,
personas y vehículos, generando alertas inmediatas ante eventos relevantes.

El presente trabajo, desarrollado en el marco de la Universidad Nacional
de Misiones, propone el diseño e implementación de un sistema de
monitoreo y alerta temprana para áreas protegidas de la región, que integra una red
de nodos de captura basados en \gls{ESP32} comunicados mediante \gls{ESP-MESH},
un servidor de procesamiento con \gls{ia} basado en \gls{speciesnet}, y
un sistema de notificaciones a través de \gls{telegram-bot}.

\section{Contexto y motivación}

Las áreas protegidas enfrentan amenazas constantes que van desde la caza
ilegal de especies en peligro hasta la intrusión de personas no autorizadas
y vehículos en zonas restringidas. Los guardaparques y administradores
de estas áreas frecuentemente carecen de los recursos humanos y tecnológicos
necesarios para mantener una vigilancia efectiva sobre extensas superficies
de terreno, muchas veces en ubicaciones remotas con acceso limitado a
infraestructura de comunicaciones.

Las cámaras trampa tradicionales, aunque útiles para la investigación
científica, no fueron diseñadas para la vigilancia en tiempo real. Las
imágenes capturadas permanecen almacenadas en tarjetas de memoria que
deben ser recolectadas físicamente, lo que implica visitas frecuentes
al campo y retrasos significativos entre la captura de un evento y su
descubrimiento. Para cuando se detecta una intrusión o un acto de \gls{caza-furtiva},
los responsables ya se encuentran lejos del área.

Es cierto que en la actualidad existen cámaras trampa más modernas que
incorporan conectividad celular o satelital, permitiendo el envío remoto
de imágenes. Sin embargo, estos dispositivos suelen tener un costo
significativamente mayor, lo que limita su adopción masiva especialmente
en países en vías de desarrollo, donde paradójicamente se concentra gran
parte de la biodiversidad mundial. Además, muchas de estas soluciones
comerciales dependen de servicios en la nube propietarios con costos
de suscripción recurrentes.

La motivación de este proyecto surge de la necesidad de transformar el
paradigma del monitoreo pasivo hacia un sistema activo de vigilancia
inteligente, pero de manera accesible y de bajo costo. Un sistema que
no solo capture imágenes, sino que las transmita en \gls{tiempo-real}, las
analice automáticamente mediante algoritmos de \gls{ia}, y genere alertas
inmediatas cuando se detecten eventos de interés, ya sea la presencia
de fauna silvestre para fines de investigación, o la detección de intrusos
para fines de seguridad. Todo esto utilizando componentes de hardware
económicos y software de \gls{codigo-abierto}.

\section{Estructura del documento}

El presente documento se organiza en diez capítulos que describen de
manera progresiva el desarrollo del sistema propuesto:

\begin{description}
	\item[Capítulo 1 - Introducción:] Presenta el contexto general del
	      proyecto, la motivación y la estructura del documento.

	\item[Capítulo 2 - Antecedentes:] Revisa trabajos relacionados,
	      soluciones comerciales existentes y el estado del arte en sistemas
	      de monitoreo de fauna y detección de intrusos.

	\item[Capítulo 3 - Planteamiento del Problema:] Describe la
	      problemática de las áreas protegidas, las limitaciones de los
	      sistemas tradicionales y la justificación del proyecto.

	\item[Capítulo 4 - Objetivos y Alcance:] Define los objetivos
	      generales y específicos, así como el alcance y las limitaciones
	      del trabajo.

	\item[Capítulo 5 - Marco Teórico:] Presenta los fundamentos teóricos
	      sobre \gls{iot}, redes \gls{mesh}, \gls{ia} aplicada a \gls{vision-computadora},
	      y las tecnologías utilizadas.

	\item[Capítulo 6 - Metodología:] Describe el enfoque metodológico,
	      las etapas de desarrollo y las herramientas empleadas.

	\item[Capítulo 7 - Diseño del Sistema:] Detalla la arquitectura
	      general, el diseño del hardware, la red \gls{mesh}, el servicio de
	      detección y el servidor de aplicación.

	\item[Capítulo 8 - Implementación:] Presenta la implementación de
	      cada componente: nodos \gls{mesh}, nodo raíz, servicio de detección
	      y servidor de aplicación.

	\item[Capítulo 9 - Pruebas y Resultados:] Documenta las pruebas
	      realizadas y analiza los resultados obtenidos en conectividad,
	      detección, rendimiento y consumo energético.

	\item[Capítulo 10 - Conclusiones:] Resume las conclusiones generales,
	      los aportes del trabajo, trabajos futuros y recomendaciones.
\end{description}

% ==============================================================================
% Capítulo 2: Antecedentes
% ==============================================================================
\chapter{Antecedentes}

En los últimos años, el campo de la \gls{ia} aplicada a la \gls{vision-computadora} ha experimentado avances significativos. El desarrollo de
arquitecturas de \glspl{cnn} cada vez más eficientes, junto con la
disponibilidad de grandes conjuntos de datos de entrenamiento, ha permitido
crear modelos capaces de detectar y clasificar objetos en imágenes con
una precisión sin precedentes \cite{schneider2018deep}. Algoritmos como
\gls{yolo} \cite{redmon2016yolo} y sus sucesivas versiones, incluyendo
\gls{yolov5} \cite{jocher2020yolov5}, han revolucionado la \gls{deteccion-objetos},
permitiendo el procesamiento en \gls{tiempo-real} incluso en dispositivos
con recursos limitados.

En el ámbito específico del monitoreo de fauna silvestre, estos avances
han dado lugar a modelos especializados como \gls{speciesnet}, desarrollado
por Google \cite{gadot2024crop}, que combina detección de objetos con
\gls{clasificacion-taxonomica} para identificar especies a partir de imágenes
de \glspl{camara-trampa}. Estos desarrollos han abierto nuevas posibilidades
para automatizar el análisis de las enormes cantidades de imágenes que
generan los sistemas de monitoreo.

Paralelamente, la tecnología de \glspl{camara-trampa} ha evolucionado desde
dispositivos autónomos que almacenan imágenes localmente, hasta sistemas
más sofisticados con conectividad celular o satelital que permiten la
transmisión remota de datos. Este capítulo examina tanto los avances en
\gls{ia} aplicada a la \gls{vision-computadora}, como la evolución de las
soluciones de monitoreo de fauna, incluyendo sistemas comerciales y
proyectos de investigación relacionados.

\section{Trabajos relacionados}

En el ámbito local, Barrero y Schmunck \cite{barrero2023microcamara} desarrollaron
en la Universidad Nacional de Misiones una microcámara de vigilancia orientada
a la protección de fauna salvaje. Este trabajo propuso el diseño de un
dispositivo compacto basado en microcontroladores capaz de capturar imágenes
en áreas naturales. El proyecto sentó las bases para el desarrollo de
soluciones de bajo costo adaptadas a las necesidades específicas de la
región misionera, demostrando la viabilidad de utilizar hardware económico
para aplicaciones de monitoreo ambiental. El presente trabajo extiende
estos conceptos incorporando comunicación en red \gls{mesh}, procesamiento
con \gls{ia} y un sistema de alertas en \gls{tiempo-real}.

En el contexto regional argentino, González et al. \cite{gonzalez2022fauna}
presentaron en el Workshop de Investigadores en Ciencias de la Computación
(WICC 2022) un sistema de \gls{ia} para la multi-clasificación de fauna
en fotografías automáticas utilizadas en investigación científica. Este
trabajo demuestra el creciente interés en la comunidad científica argentina
por aplicar técnicas de \gls{aprendizaje-automatico} al análisis de imágenes
de \glspl{camara-trampa}, estableciendo un precedente importante para
proyectos de monitoreo de fauna en el país.

En el contexto internacional, diversos trabajos han explorado el uso de
\gls{yolov5} para el análisis automatizado de imágenes de \glspl{camara-trampa}.
Abood et al. \cite{abood2023yolov5wildlife} presentaron un enfoque innovador
en la conferencia IEEE ICMNWC 2023, demostrando que los modelos de la familia
\gls{yolo} pueden adaptarse eficazmente para la detección y clasificación
de fauna silvestre. Los autores destacaron las ventajas de \gls{yolov5} en
términos de velocidad de \gls{inferencia} y precisión. De manera similar,
Njathi et al. \cite{njathi2023annotation} propusieron en IEEE AFRICON 2023
un sistema eficiente de anotación de imágenes de cámaras trampa basado en
\gls{yolov5}, enfocándose en reducir el tiempo y esfuerzo requerido para
etiquetar grandes conjuntos de datos. Ambos trabajos evidencian la idoneidad
de los modelos \gls{yolo} para aplicaciones de monitoreo de fauna que
requieren procesamiento eficiente de grandes volúmenes de imágenes.

Un antecedente particularmente relevante para el presente trabajo es el
sistema propuesto por Whytock et al. \cite{whytock2023iridium}, quienes
desarrollaron cámaras trampa con \gls{ia} capaces de enviar alertas en
\gls{tiempo-real} a través de la red satelital Iridium. El estudio, realizado
en Gabón (África Central), demostró la viabilidad de integrar procesamiento
con \gls{ia} directamente en dispositivos de campo para detectar fauna y
enviar notificaciones inmediatas a investigadores y guardaparques. Aunque
su solución utiliza conectividad satelital ---con costos operativos
significativos---, el concepto de alertas en tiempo real basadas en
detección automática constituye un pilar fundamental del sistema propuesto
en este trabajo. La diferencia principal radica en que nuestra solución
emplea redes \gls{mesh} locales y conectividad Wi-Fi/TCP, reduciendo
considerablemente los costos de operación.

Otro proyecto directamente comparable es AiCatcher, desarrollado por
Mallya \cite{mallya2019aicatcher}, que propone una cámara trampa inteligente
capaz de realizar \gls{inferencia} directamente en el dispositivo de campo.
AiCatcher utiliza una Raspberry Pi como unidad de procesamiento para ejecutar
modelos de detección en el borde (\gls{edge-computing}), y emplea módulos \gls{lora}
(Adafruit LoRa Bonnets) configurados en una red \gls{lorawan} para la transmisión
de datos, convirtiendo las señales recibidas en mensajes SMS mediante Twilio.
Si bien este enfoque es innovador, el uso de Raspberry Pi presenta un consumo
energético significativamente mayor, limitando la autonomía del dispositivo
en campo. En contraste, nuestra solución emplea microcontroladores \gls{ESP32}
de bajo consumo para la captura y transmisión, delegando el procesamiento
de \gls{ia} a un servidor remoto. Además, el uso de redes \gls{mesh} en
nuestra propuesta permite una topología de red más flexible y autoorganizada.

\section{Estado del arte}

Vélez et al. \cite{velez2022platforms} realizaron una evaluación exhaustiva
de las plataformas disponibles para el procesamiento de imágenes de
\glspl{camara-trampa} utilizando \gls{ia}. Su estudio, publicado en
\textit{Methods in Ecology and Evolution}, comparó múltiples herramientas
en términos de precisión, facilidad de uso y requisitos computacionales.
Los autores concluyeron que, si bien existen diversas opciones de
\gls{codigo-abierto} y comerciales, la elección de la plataforma adecuada
depende del contexto específico de cada proyecto, incluyendo el volumen
de imágenes, las especies objetivo y los recursos disponibles. Este análisis
proporciona un marco de referencia valioso para la selección de tecnologías
en proyectos de monitoreo de fauna.

Tabak et al. \cite{tabak2019machine} demostraron la aplicabilidad del
\gls{aprendizaje-automatico} para clasificar especies animales en imágenes de
cámaras trampa, alcanzando precisiones superiores al 90\% en la identificación
de especies comunes. Su trabajo destacó la importancia de contar con conjuntos
de datos de entrenamiento representativos de las condiciones locales para
optimizar el rendimiento de los modelos.

Por su parte, Steenweg et al. \cite{steenweg2017scaling} abordaron el desafío
de escalar las redes de \glspl{camara-trampa} para el monitoreo de biodiversidad
a nivel global. Los autores argumentaron que la integración de sensores remotos
en redes coordinadas, combinada con técnicas de análisis automatizado, representa
el futuro del monitoreo de fauna silvestre, permitiendo obtener datos de
biodiversidad a escalas espaciales y temporales sin precedentes.

Entre las plataformas más recientes, Hernández et al. \cite{hernandez2024pytorchwildlife}
presentaron PyTorch-Wildlife, un framework colaborativo de aprendizaje profundo
desarrollado por Microsoft AI for Good. Esta plataforma, similar en objetivos
a \gls{speciesnet} de Google, ofrece modelos preentrenados para la detección
y clasificación de fauna silvestre en imágenes de \glspl{camara-trampa}.
PyTorch-Wildlife se distingue por su enfoque en la facilidad de uso y la
integración con el ecosistema \gls{pytorch}, facilitando tanto el despliegue de
modelos existentes como el entrenamiento de modelos personalizados para
especies específicas. La existencia de múltiples frameworks de \gls{codigo-abierto}
respaldados por grandes empresas tecnológicas evidencia la relevancia del
problema y la madurez de las soluciones disponibles.

Cabe destacar también \gls{megadetector}, desarrollado originalmente por Microsoft
AI for Earth \cite{beery2019megadetector}, que se ha convertido en una
herramienta fundamental en el procesamiento de imágenes de \glspl{camara-trampa}.
A diferencia de los clasificadores de especies, \gls{megadetector} se especializa
en la detección genérica de animales, humanos y vehículos, funcionando como
un primer filtro que reduce drásticamente el volumen de imágenes a analizar.
Esta herramienta ha sido adoptada por más de 60 organizaciones de conservación
a nivel mundial, demostrando reducciones de hasta el 90\% en el tiempo de
procesamiento de datos. Su arquitectura modular permite integrarlo como
etapa previa a clasificadores más específicos como \gls{speciesnet}.

\section{Soluciones comerciales existentes}

En el mercado actual, existen diversas soluciones comerciales diseñadas para
el monitoreo remoto de fauna. Las \glspl{camara-trampa} con conectividad
celular, como las series REVEAL de Tactacam \cite{tactacam2024reveal} y las
líneas Flex de Spypoint \cite{spypoint2024cellular}, son las más extendidas.
Estos dispositivos permiten capturar imágenes de alta resolución y enviarlas
a través de redes LTE a una aplicación móvil propietaria. Algunas de sus
características principales incluyen disparo rápido (menor a 0.5 segundos),
visión nocturna por infrarrojos y, en modelos recientes, la capacidad de
solicitar fotografías o videos bajo demanda.

Sin embargo, estas soluciones presentan limitaciones críticas para su adopción
masiva en proyectos de conservación a gran escala o en regiones remotas. En
primer lugar, dependen enteramente de la infraestructura de red celular; en
áreas de selva densa como la de Misiones, la falta de cobertura inalámbrica
en el interior de las reservas suele ser la norma, invalidando el uso de estos
dispositivos para alertas remotas. En segundo lugar, el costo operativo es
elevado, ya que además de que cada cámara en sí tiene un costo de adquisición
significativo, cada dispositivo requiere un plan de datos independiente, con
suscripciones mensuales que pueden oscilar entre los 5 y 15 dólares por unidad
\cite{tactacam2024reveal, spypoint2024cellular}.

En cuanto a la gestión de datos, plataformas como Wildlife Insights
\cite{wildlifeinsights2024} ofrecen una infraestructura robusta basada en la
nube para el almacenamiento y análisis de imágenes mediante \gls{ia}. Esta
herramienta permite filtrar imágenes vacías automáticamente y realizar la
\gls{clasificacion-taxonomica} de especies, facilitando la colaboración entre
investigadores. No obstante, Wildlife Insights está orientada principalmente al
procesamiento posterior (post-hoc) de los datos recolectados manualmente de las
tarjetas SD, y no está diseñada para la vigilancia y respuesta inmediata ante
eventos de intrusión o \gls{caza-furtiva} en tiempo real.

\section{Análisis comparativo}

A fin de situar las necesidades identificadas frente a las alternativas y
trabajos discutidos anteriormente, se presenta a continuación un análisis
comparativo que resume las principales diferencias técnicas y operativas. En la
tabla \ref{tab:comparativa-sistemas} se contrastan las características de las
categorías de soluciones analizadas: cámaras trampa tradicionales, soluciones
comerciales celulares \cite{tactacam2024reveal, spypoint2024cellular}, el
sistema satelital de Whytock et al. \cite{whytock2023iridium} y el enfoque
basado en \gls{edge-computing} de Mallya (AiCatcher) \cite{mallya2019aicatcher}.

\begin{table}[ht]
	\centering
	\caption{Comparativa de soluciones de monitoreo y trabajos relacionados.}
	\label{tab:comparativa-sistemas}
	\begin{tblr}{
			width = \textwidth,
			colspec = {X[2.2,l] *{4}{X[c]}},
			hlines,
			vlines,
			row{1} = {font=\bfseries},
		}
		Característica               & Tradicional & Celular \cite{tactacam2024reveal} & Satelital \cite{whytock2023iridium} & \gls{lora} \cite{mallya2019aicatcher} \\
		Alertas en \gls{tiempo-real} & No          & Sí                                & Sí                                  & Sí                                    \\
		Infraestructura requerida    & Ninguna     & Operador Celular                  & Red Satelital                       & Puerta de enlace \gls{lora}           \\
		Procesamiento de \gls{ia}    & Post-hoc    & No / Limitado                     & Edge (Hardware Dedicado)            & Edge (Raspberry Pi)                   \\
		Costo de adquisición         & Bajo-Medio  & Medio-Alto                        & Muy Alto                            & Medio-Alto                            \\
		Costo Operativo              & Bajo        & Alto (Plan mensual)               & Muy Alto (Iridium)                  & Bajo (SMS/Twilio)                     \\
		Autonomía energética         & Muy Alta    & Media                             & Media-Baja                          & Baja (Raspberry Pi)                   \\
		Escalabilidad                & Laboriosa   & Por suscripción                   & Costo-prohibitiva                   & Media (\gls{lora})                    \\
		Dependencia de terceros      & No          & Muy Alta                          & Muy Alta                            & Media (Twilio)                        \\
	\end{tblr}
\end{table}

Como se desprende de la comparación, si bien existen soluciones de vanguardia,
el acceso a alertas inmediatas suele estar condicionado por elevados costos
operativos o dependencia de infraestructura de terceros (celular o satelital).
Los trabajos de investigación como AiCatcher \cite{mallya2019aicatcher} y el
sistema de Whytock et al. \cite{whytock2023iridium} demuestran la viabilidad de
la \gls{ia} para el filtrado en tiempo real, pero enfrentan desafíos en cuanto
a consumo energético y costos de transmisión.

En este sentido, se identifica una oportunidad para un sistema que combine la
eficiencia en el consumo de los microcontroladores \gls{ESP32} con la
capacidad de extensión de cobertura de las redes \gls{mesh}, permitiendo un
monitoreo inteligente y de bajo costo operativo adaptado a las condiciones
reales observadas en la \gls{selva-misionera}.

% ==============================================================================
% Capítulo 3: Planteamiento del Problema
% ==============================================================================
\chapter{Planteamiento del Problema}

\section{Contexto regional}

La provincia de Misiones, ubicada en el extremo noreste de Argentina, alberga
uno de los ecosistemas más diversos y amenazados del continente \cite{wwf2024atlantic}.
El marco geográfico de este proyecto se sitúa en el corazón de la \gls{selva-paranaense},
una región que requiere estrategias de conservación urgentes y tecnificadas.

\subsection{La Selva Misionera y el Bosque Atlántico}

La \gls{selva-misionera} representa el remanente continuo más extenso del
\gls{bosque-atlantico} ---también conocido como \gls{mata-atlantica}--- en
el Cono Sur. Originalmente, este bioma cubría aproximadamente 1.3 millones
de kilómetros cuadrados, extendiéndose desde la costa atlántica de Brasil
(92\% de su extensión) hasta el este de Paraguay (6\%) y el noreste de
Argentina (2\%) \cite{ribeiro2009atlantic}.

Sin embargo, debido a siglos de expansión agrícola, ganadera y forestal,
hoy solo sobrevive entre el 12\% y 17\% de su extensión original, lo que
lo convierte en uno de los \gls{hotspot-biodiversidad} más amenazados
del planeta \cite{wwf2024atlantic}. Se estima que el \gls{bosque-atlantico}
ha perdido aproximadamente el 88\% de su cobertura vegetal nativa, siendo
reemplazado por paisajes dominados por agricultura, pasturas y áreas urbanas.

Misiones conserva cerca de 1.1 millones de hectáreas de este bosque através
del \gls{corredor-verde}, un sistema de áreas protegidas interconectadas que
posiciona a la provincia como un refugio crítico para la biodiversidad
regional y global \cite{dibitetti2003yaguarete}.

\subsection{Biodiversidad y especies en peligro}

Esta región es hogar de una concentración excepcional de biodiversidad:
a pesar de ocupar menos del 0.5\% del territorio argentino, la \gls{selva-misionera}
alberga más del 50\% de la biodiversidad del país \cite{dibitetti2003yaguarete}.
Se estima que contiene alrededor de 3\,000 especies de plantas vasculares,
554 especies de aves, 120 de mamíferos, 79 de reptiles y 55 de anfibios.

Entre las especies más emblemáticas se encuentra el yaguareté (\textit{Panthera onca}),
declarado \gls{monumento-natural} provincial en 1988 y nacional en 2001. Según
el último censo binacional realizado en 2024, se estima una población de
aproximadamente 84 individuos en el \gls{corredor-verde} entre Argentina y
Brasil, con un rango estimado entre 64 y 110 ejemplares \cite{vidasilvestre2024censo}.
Esta cifra representa casi la mitad de los menos de 250 yaguaretés adultos
que se estima sobreviven en todo el territorio argentino.

Otras especies de importancia para la conservación incluyen el tapir
(\textit{Tapirus terrestris}), el oso hormiguero gigante (\textit{Myrmecophaga tridactyla}),
la yacutinga (\textit{Aburria jacutinga}), el mono caí (\textit{Sapajus nigritus})
y el águila harpía (\textit{Harpia harpyja}). La presencia de estos grandes
vertebrados es un indicador clave del estado de salud del ecosistema, pero su
monitoreo en densas zonas de selva subtropical es extremadamente complejo.

\subsection{Importancia ecológica de la región}

La importancia de la \gls{selva-misionera} trasciende la mera preservación de
especies; actúa como regulador climático, protector de cuencas hídricas y
proveedor de \glspl{servicios-ecosistemicos} esenciales para la región, incluyendo
la regulación del ciclo hidrológico, el secuestro de carbono y la protección
contra la erosión del suelo.

El marco legal para la protección de estos ecosistemas está dado por la
Ley 26.331 de Presupuestos Mínimos de Protección Ambiental de los Bosques
Nativos \cite{ley26331bosques}, que establece un sistema de ordenamiento
territorial basado en tres categorías de conservación:

\begin{description}
	\item[Categoría I (Rojo):] Sectores de muy alto valor de conservación que
	      no deben transformarse. En Misiones, gran parte del \gls{corredor-verde}
	      se encuentra clasificado en esta categoría.
	\item[Categoría II (Amarillo):] Sectores de mediano valor de conservación
	      donde se permite el aprovechamiento sostenible y la restauración.
	\item[Categoría III (Verde):] Sectores de bajo valor de conservación que
	      pueden transformarse parcialmente.
\end{description}

La provincia de Misiones gestiona actualmente 104 áreas naturales protegidas
que suman aproximadamente 1.1 millones de hectáreas, lo que impone la necesidad
de una vigilancia constante para evitar la degradación de estas áreas protegidas
frente a actividades ilícitas como la \gls{caza-furtiva} y la tala clandestina.

\section{Problemáticas de conservación en la región}

A pesar de los esfuerzos institucionales y del marco legal establecido por la
Ley 26.331 \cite{ley26331bosques}, las áreas protegidas de Misiones enfrentan
amenazas persistentes que comprometen la integridad de sus ecosistemas.

\subsection{Deforestación y fragmentación del hábitat}

La pérdida de cobertura boscosa por actividades agrícolas no autorizadas y la
extracción ilegal de madera noble continúan siendo problemas recurrentes. Un
estudio de la Facultad de Agronomía de la Universidad de Buenos Aires reveló
que entre 1990 y 2020 se perdieron aproximadamente 130\,000 hectáreas de bosque
nativo solo en el \gls{corredor-verde}, lo que representa un 13\% del área
original \cite{fauba2024corredorverde}. El análisis indica que el 77\% de la
deforestación ocurre en parcelas menores a 50 hectáreas, frecuentemente
asociadas a ocupaciones espontáneas para cultivos de subsistencia.

Si bien las políticas de control han mostrado resultados positivos ---con una
reducción del 18\% en la deforestación durante 2025, alcanzando 4\,118 hectáreas
anuales frente al promedio histórico de 5\,000 hectáreas
\cite{ecologiamisiones2025deforestacion}---, la \gls{fragmentacion-habitat}
sigue obligando a las especies de gran tamaño, como el yaguareté, a desplazarse
por áreas no protegidas. Esto aumenta el riesgo de conflictos con humanos,
atropellamientos en rutas, y reduce la viabilidad genética de las poblaciones
aisladas.

\subsection{Caza furtiva y tráfico de fauna}

La \gls{caza-furtiva} representa uno de los mayores desafíos para la
conservación en la región. A pesar de la prohibición total de la caza en la
provincia, la incursión de cazadores en el interior de parques provinciales
y reservas privadas es frecuente \cite{ecologiamisiones2024cazafurtiva}. El
problema presenta dos dimensiones principales: una cultural, llevada a cabo
por residentes locales como actividad de subsistencia, y otra de carácter
económico, impulsada por grupos organizados que buscan especies valiosas
para el \gls{trafico-fauna}.

Los puntos más críticos se concentran en las zonas fronterizas con Brasil,
especialmente en áreas como la \gls{reserva-biosfera} Yabotí y los parques
provinciales Piñalito, Urugua-í y Horacio Foerster. Entre las especies más
afectadas se encuentran el tapir (\textit{Tapirus terrestris}), la paca
(\textit{Cuniculus paca}), corzuelas, y aves como tucanes, guacamayos y loros.
Los cazadores utilizan técnicas de acecho, trampas y cebaderos que no solo
afectan a las especies objetivo, sino que degradan la fauna en su totalidad
y ponen en riesgo la seguridad de los guardaparques durante los operativos
de control.

\subsection{Intrusión en áreas protegidas}

La falta de un control perimetral efectivo en las más de 106 áreas protegidas
de la provincia permite la entrada de personas y vehículos no autorizados para
actividades ilícitas. Entre las más frecuentes se encuentran:

\begin{itemize}
	\item Pesca ilegal en cursos de agua dentro de reservas.
	\item Desmonte encubierto para expansión de cultivos.
	\item Instalación de campamentos de caza con infraestructura permanente.
\end{itemize}

Sin un sistema de \gls{alerta-temprana}, estas intrusiones frecuentemente solo se
descubren a posteriori durante patrullajes de rutina, cuando el daño ambiental
ya ha sido perpetrado y los responsables se encuentran lejos del área.

\subsection{Desafíos de la vigilancia en el terreno}

La vigilancia efectiva de áreas protegidas en la \gls{selva-misionera} enfrenta
obstáculos inherentes a las características del terreno y la extensión de las
superficies a cubrir. Con aproximadamente 780\,000 hectáreas distribuidas en
más de 106 áreas protegidas, cualquier estrategia de monitoreo debe contemplar
las siguientes limitaciones:

\begin{description}
	\item[Extensión y accesibilidad:] Las vastas superficies boscosas presentan
	      terreno accidentado, cursos de agua y vegetación densa que dificultan
	      el desplazamiento y limitan el alcance de las patrullas terrestres.
	\item[Cobertura de comunicaciones:] La ausencia de infraestructura celular
	      en zonas profundas de selva impide la comunicación en \gls{tiempo-real}
	      y la coordinación de respuestas ante eventos detectados.
	\item[Latencia en la detección:] El uso de \glspl{camara-trampa} tradicionales,
	      si bien genera datos valiosos para investigación, almacena las imágenes
	      localmente en tarjetas de memoria que deben ser retiradas físicamente,
	      introduciendo demoras de semanas o meses entre la captura y el análisis.
	\item[Volumen de datos:] Las cámaras convencionales generan grandes cantidades
	      de imágenes, muchas disparadas por movimiento de vegetación o fauna no
	      relevante, requiriendo un esfuerzo manual significativo para su revisión.
\end{description}

Estas limitaciones evidencian la necesidad de sistemas tecnológicos capaces de
transmitir alertas en \gls{tiempo-real}, extender la cobertura de comunicaciones
mediante redes autoorganizadas, y filtrar automáticamente las imágenes relevantes
mediante técnicas de \gls{ia}.

\section{Sistemas de monitoreo actuales}

Actualmente, el monitoreo biológico y de vigilancia en las áreas protegidas de
Misiones emplea una combinación de métodos tradicionales que, si bien han
demostrado ser efectivos para la investigación científica, presentan brechas
operativas significativas cuando se aplican a la seguridad ambiental y la
detección de actividades ilícitas.

\subsection{Cámaras trampa convencionales}

El uso de \glspl{camara-trampa} pasivas constituye el estándar de referencia
para el monitoreo de fauna silvestre a nivel mundial \cite{steenweg2017scaling}.
Estos dispositivos se instalan en puntos estratégicos ---senderos de fauna,
fuentes de agua, zonas de paso--- y capturan imágenes automáticamente mediante
sensores de calor y movimiento (\gls{sensor-pir}).

Las \glspl{camara-trampa} tradicionales presentan las siguientes características
operativas:

\begin{itemize}
	\item \textbf{Autonomía energética:} Funcionan con baterías AA o paneles
	      solares, permitiendo operación autónoma durante meses.
	\item \textbf{Almacenamiento local:} Las imágenes se guardan en tarjetas
	      SD que deben ser retiradas físicamente para su análisis.
	\item \textbf{Activación pasiva:} El sensor \gls{pir} detecta cambios de
	      temperatura asociados al movimiento de animales o personas.
	\item \textbf{Operación silenciosa:} Los modelos con flash infrarrojo
	      (\gls{ir}) permiten capturas nocturnas sin alertar a la fauna.
\end{itemize}

En Misiones, estas cámaras son utilizadas tanto por organismos gubernamentales
como por organizaciones de conservación para el monitoreo del yaguareté,
realizar censos de fauna, y documentar la biodiversidad de las reservas
\cite{vidasilvestre2024censo}.

\subsection{Patrullajes terrestres}

Complementariamente, el Cuerpo de Guardaparques realiza patrullajes periódicos
a pie, en vehículo o a caballo. Estos recorridos permiten detectar indicios
de actividad ilícita (campamentos, trampas, rastros de tala) y establecer
presencia disuasoria en las áreas protegidas.

Sin embargo, la efectividad de los patrullajes está limitada por factores
logísticos: la densidad de la vegetación restringe la visibilidad, los
desplazamientos consumen tiempo considerable, y la cobertura territorial
depende directamente de los recursos humanos disponibles.

\subsection{Brechas tecnológicas identificadas}

El análisis de los sistemas actuales revela brechas críticas que limitan
la capacidad de respuesta ante eventos de interés:

\begin{description}
	\item[Latencia en la información:] Las imágenes capturadas por cámaras
	      tradicionales permanecen almacenadas localmente durante semanas o
	      meses hasta su recolección. Un evento de \gls{caza-furtiva} registrado
	      por una cámara solo será conocido mucho después de ocurrido,
	      impidiendo cualquier acción disuasoria o legal inmediata.
	\item[Clasificación manual:] El alto volumen de imágenes capturadas
	      ---frecuentemente disparadas por movimiento de vegetación, aves, o
	      pequeños mamíferos no relevantes--- demanda cientos de horas-hombre
	      para su revisión y clasificación manual \cite{tabak2019machine}.
	\item[Ausencia de alertas:] Los sistemas actuales no generan notificaciones
	      en \gls{tiempo-real}. La información fluye de manera unidireccional
	      desde el campo hacia los centros de análisis, sin retroalimentación
	      inmediata.
	\item[Conectividad limitada:] Las soluciones comerciales con transmisión
	      celular, aunque existen en el mercado \cite{tactacam2024reveal,
		      spypoint2024cellular}, resultan inviables en las zonas interiores de
	      la selva donde no existe cobertura de red móvil, además de representar
	      costos operativos elevados para despliegues a escala.
\end{description}

Estas brechas fundamentan la necesidad de desarrollar un sistema que combine
la robustez de las cámaras de campo con capacidades de transmisión en red,
procesamiento automático mediante \gls{ia}, y generación de alertas en
\gls{tiempo-real}.

\section{Justificación del proyecto}

El análisis de las problemáticas de conservación y las limitaciones de los
sistemas de monitoreo actuales permite identificar una brecha tecnológica
que este proyecto propone abordar.

\subsection{Urgencia de la problemática}

La \gls{selva-misionera} representa el último remanente continuo del
\gls{bosque-atlantico} en Argentina, albergando más del 50\% de la biodiversidad
del país en menos del 0.5\% de su territorio \cite{dibitetti2003yaguarete}. La
presión sobre este ecosistema es constante: entre 1990 y 2020 se perdieron
130\,000 hectáreas de bosque nativo solo en el \gls{corredor-verde}
\cite{fauba2024corredorverde}, mientras que especies emblemáticas como el
yaguareté mantienen poblaciones críticamente bajas ---aproximadamente 84
individuos según el último censo \cite{vidasilvestre2024censo}.

La \gls{caza-furtiva}, la tala ilegal y la intrusión en áreas protegidas
continúan siendo amenazas activas que requieren respuestas inmediatas. Sin
embargo, los sistemas de monitoreo actuales operan con latencias de semanas
o meses, imposibilitando cualquier acción preventiva o disuasoria.

\subsection{Oportunidad tecnológica}

La convergencia de tres desarrollos tecnológicos recientes crea una oportunidad
única para abordar esta problemática:

\begin{description}
	\item[Microcontroladores de bajo costo:] Plataformas como el \gls{ESP32}
	      ofrecen capacidades de procesamiento, conectividad Wi-Fi y gestión
	      de energía a costos inferiores a los 10 dólares por unidad, permitiendo
	      despliegues a escala con presupuestos limitados.
	\item[Redes mesh autoorganizadas:] Protocolos como \gls{ESP-MESH} permiten
	      extender la conectividad en áreas sin infraestructura celular,
	      creando redes que se adaptan dinámicamente a la topología del terreno
	      y toleran fallos de nodos individuales.
	\item[Inteligencia artificial para visión:] Modelos de \gls{codigo-abierto}
	      como \gls{speciesnet} \cite{gadot2024crop} y \gls{megadetector}
	      \cite{beery2019megadetector} permiten clasificar automáticamente
	      imágenes de \glspl{camara-trampa}, distinguiendo fauna silvestre
	      de personas y vehículos con alta precisión.
\end{description}

\subsection{Requisitos del sistema propuesto}

Dada la crítica situación de biodiversidad en Misiones y las brechas identificadas
en los sistemas actuales, se justifica el desarrollo de una solución tecnológica
que cumpla con los siguientes requisitos:

\begin{enumerate}
	\item \textbf{Bajo costo de implementación:} Permitiendo un despliegue
	      distribuido con presupuesto limitado, utilizando hardware basado en
	      \gls{ESP32} y software de \gls{codigo-abierto}.
	\item \textbf{Conectividad independiente:} Empleando redes \gls{mesh} que
	      no requieran infraestructura celular preexistente, adaptándose a la
	      topología irregular de la selva subtropical.
	\item \textbf{Procesamiento inteligente:} Integrando modelos de \gls{ia}
	      como \gls{speciesnet} para filtrar y clasificar imágenes automáticamente,
	      reduciendo el volumen de datos a revisar manualmente.
	\item \textbf{Alertas en tiempo real:} Generando notificaciones inmediatas
	      ante la detección de eventos críticos, ya sea la presencia de fauna
	      en peligro para fines de investigación, o la detección de intrusos
	      para fines de seguridad.
\end{enumerate}

Este proyecto no solo representa un aporte técnico en el área de redes de
sensores e \gls{ia} aplicada a la conservación, sino que constituye una
herramienta de aplicación directa para fortalecer la vigilancia de las áreas
protegidas de la región, democratizando el acceso a tecnologías de monitoreo
inteligente que tradicionalmente han estado reservadas a instituciones con
mayores recursos.

% ==============================================================================
% Capítulo 4: Objetivos y Alcance
% ==============================================================================
\chapter{Objetivos y Alcance}

\section{Objetivo general}

\section{Objetivos específicos}

\section{Alcance del proyecto}

\section{Limitaciones}

% ==============================================================================
% Capítulo 5: Marco Teórico
% ==============================================================================
\chapter{Marco Teórico}

\section{Internet de las Cosas (IoT)}

El \gls{iot} se define como la red de objetos físicos que incorporan sensores,
software y otras tecnologías con el fin de conectar e intercambiar datos con
otros dispositivos y sistemas a través de internet. En el contexto de la
conservación ambiental, el IoT permite la creación de redes de monitoreo
autónomas que pueden operar en ubicaciones remotas con mínima intervención
humana.

\subsection{Arquitecturas IoT}

Una arquitectura típica de IoT se compone de tres capas fundamentales:
\begin{enumerate}
	\item \textbf{Capa de Percepción:} Formada por los sensores y actuadores
	      que interactúan directamente con el entorno físico.
	\item \textbf{Capa de Red:} Encargada de la transmisión de datos desde los
	      nodos sensores hacia los sistemas de procesamiento.
	\item \textbf{Capa de Aplicación:} Donde los datos son procesados,
	      almacenados y presentados al usuario final.
\end{enumerate}
El sistema desarrollado en esta tesis abarca estas tres capas, utilizando
nodos \gls{ESP32} en la percepción, \gls{ESP-MESH} en la red y un servidor
\gls{django} en la aplicación.

\subsection{Protocolos de comunicación inalámbrica}

La elección del protocolo de comunicación es crítica en entornos rurales.
Mientras que protocolos como Bluetooth y ZigBee son ideales para corto alcance
y baja potencia, y LoRa es excelente para muy largo alcance con bajo ancho de
banda, las redes Wi-Fi (802.11) ofrecen un equilibrio adecuado para la
transmisión de imágenes si se gestionan adecuadamente mediante una topología
\gls{mesh}, permitiendo extender la cobertura sin depender de una única
estación base.

\section{Redes Mesh}

Una red de malla o \gls{mesh} es una topología en la que los nodos se conectan
entre sí de forma dinámica y no jerárquica, cooperando para propagar los datos
a través de la red. Esta arquitectura es especialmente robusta frente a fallos
de nodos individuales, ya que la red puede reconfigurarse automáticamente para
encontrar nuevas rutas.

\subsection{Topologías de red}

A diferencia de las topologías en estrella (comunes en redes Wi-Fi domésticas
donde todos los dispositivos dependen de un mismo Punto de Acceso), las redes
\gls{mesh} permiten que cada \gls{nodo} actúe también como un repetidor. Esto
facilita el despliegue en terrenos difíciles o boscosos, donde los obstáculos
físicos limitan la línea de vista. La red se organiza en una estructura de
árbol donde existe un \gls{nodo-raiz} que actúa como puerta de enlace, y
nodos intermedios que transportan los datos de los nodos hoja.

\subsection{ESP-MESH y Mwifi}

\gls{ESP-MESH} es el protocolo propietario de Espressif basado en Wi-Fi que
permite conectar hasta miles de dispositivos \gls{ESP32} en una sola red. Su
principal ventaja es la capacidad de autoorganización y autocuración: si un
nodo intermedio falla, sus nodos descendientes buscan automáticamente un nuevo
padre. Por otro lado, \gls{mwifi} es un componente del \gls{ESP-MDF} que
simplifica el desarrollo de aplicaciones mesh al proporcionar una capa de
abstracción de alto nivel, facilitando la transmisión de paquetes TCP/UDP
y la gestión de la topología desde el código de usuario.

\section{Inteligencia Artificial aplicada a visión por computadora}

La \gls{vision-computadora} busca emular la capacidad humana de interpretar
imágenes digitales. Los avances en este campo están impulsados por el
\gls{aprendizaje-automatico} y, más específicamente, por el aprendizaje profundo
(deep learning).

\subsection{Redes neuronales convolucionales (CNN)}

Las \glspl{cnn} son arquitecturas especializadas en procesar datos con una
topología de cuadrícula, como las imágenes. Funcionan mediante la aplicación de
filtros o núcleos que extraen características fundamentales (bordes, texturas,
formas compleja) en distintas capas jerárquicas. Esta capacidad de extracción
automática de características las hace ideales para la clasificación de fauna y
la detección de intrusos.

\subsection{Detección de objetos con YOLO}

\gls{yolo} es uno de los algoritmos de detección de objetos más populares
debido a su velocidad y precisión. A diferencia de otros métodos que analizan
una imagen en múltiples pasadas, YOLO (\textit{You Only Look Once}) trata la
detección como un problema de regresión único, prediciendo las \glspl{bounding-box}
y las probabilidades de clase simultáneamente para toda la imagen. El uso de
\gls{yolov5} en este proyecto se justifica por su excelente equilibrio entre
rendimiento computacional y exactitud, permitiendo procesar múltiples flujos de
imágenes en el servidor.

\subsection{SpeciesNet de Google}

\gls{speciesnet} es un sistema de \gls{ia} de \gls{codigo-abierto} desarrollado
por Google \cite{gadot2024crop} diseñado específicamente para el análisis de
fotos de \glspl{camara-trampa}. Utiliza modelos preentrenados potentes para
identificar cientos de especies globales, permitiendo realizar la
\gls{clasificacion-taxonomica} de manera automatizada. En este sistema,
SpeciesNet actúa como el núcleo de procesamiento que valida si una imagen
capturada por los nodos contiene fauna de interés, personas o vehículos.

\section{Tecnologías de desarrollo}

Para la implementación del sistema se han seleccionado herramientas que
garantizan la modularidad, facilidad de mantenimiento e interoperabilidad.

\subsection{Microcontroladores ESP32}

El \gls{ESP32} es un System on Chip (SoC) de bajo costo y bajo consumo con Wi-Fi
y Bluetooth integrados. Pertenece a la familia de microcontroladores de
Espressif Systems y destaca por su procesador de doble núcleo, rica variedad de
periféricos y su idoneidad para aplicaciones de IoT industriales y ambientales.

\subsection{ESP-IDF y ESP-MDF}

El \gls{ESP-IDF} es el entorno de desarrollo oficial para el ESP32, basado en
FreeRTOS, que permite un control preciso sobre el hardware y la gestión de la
energía. Sobre este entorno se construye el \gls{ESP-MDF} (Mesh Development
Framework), que proporciona las herramientas específicas para la creación y
gestión de redes mesh, incluyendo capacidades de actualización remota (OTA).

\subsection{Contenedorización con Docker}

\gls{docker} es una plataforma que permite empaquetar software en unidades
estandarizadas llamadas contenedores, las cuales incluyen todo lo necesario para
que la aplicación se ejecute correctamente. El uso de \gls{docker} y
\gls{docker-compose} en este proyecto facilita el despliegue del servidor de
procesamiento (\gls{speciesnet}, base de datos y backend) en cualquier máquina,
asegurando que el entorno de ejecución sea idéntico al de desarrollo.

\subsection{Framework Django}

\gls{django} es un framework web de alto nivel escrito en Python que fomenta el
desarrollo rápido y un diseño limpio. Se utiliza para construir el servidor
central que recibe las imágenes de la red mesh, interactúa con el sistema de
\gls{ia} y gestiona la base de datos \gls{postgresql}. Su arquitectura MVT
(Modelo-Vista-Plantilla) permite una separación clara entre la lógica de datos
y la interfaz de usuario.

\section{Sensores y actuadores}

Los componentes físicos periféricos permiten la interacción del nodo con el
entorno, detectando eventos y capturando información visual.

\subsection{Sensores de movimiento PIR}

El \gls{sensor-pir} (Passive Infrared) detecta movimiento mediante la medición
de los cambios en los niveles de radiación infrarroja emitida por objetos
cercanos. En este proyecto, el sensor PIR actúa como el disparador del sistema:
cuando se detecta calor en movimiento (característico de humanos o animales), el
microcontrolador sale de su modo de bajo consumo y activa la cámara para la
captura.

\subsection{Módulos de cámara}

Se utilizan módulos de cámara compatibles con el ESP32 (como la OV2640), que
permiten capturar imágenes en resoluciones de hasta 2 megapíxeles. Estos módulos
están integrados en placas de desarrollo como la ESP32-CAM, proporcionando una
solución compacta para la adquisición visionaria de datos con un consumo
controlado.

\section{Diseño y fabricación digital}

La protección física del hardware es tan importante como el software en
entornos de selva subtropical.

\subsection{Modelado 3D}

El diseño de la \gls{carcasa} se realiza mediante herramientas de \gls{cad},
permitiendo crear estructuras precisas que protejan los componentes de la
humedad y la lluvia, a la vez que proporcionan aperturas para la lente de la
cámara y el ángulo de visión del sensor PIR.

\subsection{Fabricación aditiva}

La \gls{impresion-3d} es la tecnología utilizada para materializar los diseños
CAD. Permite iterar rápidamente sobre prototipos y fabricar carcasas robustas
utilizando filamentos resistentes como el PETG o el ASA, que soportan mejor la
radiación UV y las altas temperaturas exteriores en comparación con el PLA.

% ==============================================================================
% Capítulo 6: Metodología
% ==============================================================================
\chapter{Metodología}

\section{Enfoque metodológico}

\section{Etapas del desarrollo}

\section{Herramientas y tecnologías utilizadas}

\section{Métricas de evaluación}

% ==============================================================================
% Capítulo 7: Diseño del Sistema
% ==============================================================================
\chapter{Diseño del Sistema}

\section{Arquitectura general}

\section{Diseño del hardware}
\subsection{Selección de componentes}
\subsection{Nodo de captura con cámara}
\subsection{Sensor de movimiento PIR}
\subsection{Carcasa para impresión 3D}
\subsection{Nodo raíz}
\subsection{Alimentación y consumo energético}

\section{Diseño de la red mesh}
\subsection{Topología de la red}
\subsection{Protocolo de comunicación}
\subsection{Formato de datos}

\section{Diseño del servicio de detección}
\subsection{Servidor de inferencia con SpeciesNet}
\subsection{Detección de animales, humanos y vehículos}
\subsection{Anotación de imágenes con bounding boxes}

\section{Diseño del servidor de aplicación}
\subsection{Arquitectura de servicios}
\subsection{Gestión de imágenes}
\subsection{Interfaz web}
\subsection{Bot de Telegram y sistema de alertas}

% ==============================================================================
% Capítulo 8: Implementación
% ==============================================================================
\chapter{Implementación}

\section{Nodo mesh (mesh-node)}
\subsection{Firmware del nodo de captura}
\subsection{Integración del sensor PIR}
\subsection{Captura de imágenes}
\subsection{Compresión y transmisión}
\subsection{Fabricación de la carcasa 3D}

\section{Nodo raíz (root-node)}
\subsection{Firmware del nodo raíz}
\subsection{Conexión con servidor TCP}
\subsection{Gestión de la red mesh}

\section{Servicio de detección (wildlife-detection)}
\subsection{Contenedor Docker con SpeciesNet}
\subsection{API de inferencia con LitServe}
\subsection{Procesamiento de imágenes}

\section{Servidor de aplicación (server)}
\subsection{Aplicación Django}
\subsection{Integración con SpeciesNet}
\subsection{Bot de Telegram y sistema de alertas}
\subsection{Base de datos PostgreSQL}
\subsection{Despliegue con Docker Compose}

% ==============================================================================
% Capítulo 9: Pruebas y Resultados
% ==============================================================================
\chapter{Pruebas y Resultados}

\section{Ambiente de pruebas}

\section{Pruebas de conectividad y red mesh}
\subsection{Alcance de la red}
\subsection{Latencia de transmisión}
\subsection{Estabilidad de la conexión}

\section{Pruebas de detección}
\subsection{Detección de fauna silvestre}
\subsection{Detección de humanos}
\subsection{Detección de vehículos}

\section{Evaluación del modelo de IA}
\subsection{Precisión y recall}
\subsection{Tiempo de inferencia}

\section{Pruebas de consumo energético}

\section{Pruebas del sistema de alertas}
\subsection{Tiempo de respuesta}

\section{Análisis de resultados}

% ==============================================================================
% Capítulo 10: Conclusiones
% ==============================================================================
\chapter{Conclusiones}

\section{Conclusiones generales}

\section{Aportes del trabajo}

\section{Trabajos futuros}

\section{Recomendaciones}

% ==============================================================================
% Referencias
% ==============================================================================
\printbibliography[heading=bibnumbered]

% ==============================================================================
% Anexos
% ==============================================================================
\appendix

\chapter{Esquemáticos del hardware}

\chapter{Código fuente relevante}
\section{Firmware del nodo mesh}
\section{Firmware del nodo raíz}
\section{Servidor de detección}
\section{Aplicación Django}

\chapter{Manual de instalación y configuración}
\section{Configuración del firmware}
\section{Despliegue del servidor}
\section{Configuración del bot de Telegram}

\chapter{Manual de usuario}

\chapter{Especificaciones técnicas}

\chapter{Análisis de viabilidad económica}

\end{document}