% Tipo de documento y márgenes
\documentclass[a4paper, 12pt]{report}
\usepackage[top=3cm, bottom=3cm, left = 2cm, right = 2cm]{geometry}

% Codificación y lenguaje
\usepackage[utf8]{inputenc}
\usepackage{csquotes}
\usepackage[spanish]{babel}

% Imágenes
\usepackage{graphicx}
\usepackage{subcaption}

% Bibliografía
\usepackage[nottoc]{tocbibind}
\usepackage[style=ieee, backend=biber]{biblatex}
\addbibresource{referencias.bib}

% Links
\usepackage{hyperref}
\hypersetup{colorlinks=true,linkcolor=black,citecolor=black,urlcolor=blue}

% Glosario
\usepackage{datatool}[=v2.32]
\usepackage[toc,acronym]{glossaries}
\setacronymstyle{long-short}
\makenoidxglossaries
\loadglsentries{glosario}

\usepackage[locale=DE]{siunitx}

\title{Sistema de Monitoreo y Alerta Temprana basado en Inteligencia Artificial
para Áreas Protegidas}
\author{Autores:\\Fabrizio Martin Contigiani\\Gabriel Orlando Da
Silva Schmies \\\\Tutor:\\Dr. Ing. Sergio Eduardo Moya}
\date{\today}

\begin{document}

\maketitle

\begin{abstract}
	El presente trabajo describe el diseño e implementación de un sistema
	de monitoreo y alerta temprana para áreas protegidas, que combina
	tecnologías de \gls{iot} con \gls{ia}
	para la detección automática de fauna silvestre, personas y vehículos.

	El sistema está compuesto por una red de \glspl{nodo} de captura basados en
	microcontroladores \gls{ESP32} equipados con cámaras, los cuales se comunican
	mediante el protocolo \gls{ESP-MESH} para transmitir imágenes hacia un
	\gls{nodo-raiz}. Este nodo actúa como puerta de enlace, reenviando las imágenes
	a un servidor remoto a través de \gls{tcp}/IP.

	En el servidor, las imágenes son procesadas por un servicio de
	\gls{inferencia} basado en \gls{speciesnet}, un modelo de detección de objetos
	desarrollado por Google que utiliza \gls{yolov5}. Este modelo permite
	identificar y clasificar especies de fauna silvestre, así como detectar
	la presencia de humanos y vehículos, generando alertas automáticas
	ante posibles intrusiones.

	La arquitectura del servidor incluye una aplicación web desarrollada
	en \gls{django} para la gestión de imágenes, un \gls{telegram-bot} para el
	envío de notificaciones en tiempo real, y una base de datos \gls{postgresql}
	para el almacenamiento persistente. Todo el sistema está
	contenedorizado mediante \gls{docker} para facilitar su despliegue.

	Los resultados demuestran la viabilidad de implementar un sistema de
	vigilancia inteligente de bajo costo para áreas protegidas, capaz de
	operar de manera autónoma y alertar a los administradores ante
	eventos relevantes.

	\vspace{2em}
	\noindent\textbf{Palabras Clave} - Cámaras Trampa, Internet de las Cosas,
	Inteligencia Artificial, Monitoreo de Fauna, Detección de Intrusos,
	ESP-MESH, SpeciesNet, YOLO
\end{abstract}

\tableofcontents

\listoffigures

\listoftables

\printnoidxglossary[type=main]
\printnoidxglossary[type=\acronymtype]

% ==============================================================================
% Capítulo 1: Introducción
% ==============================================================================
\chapter{Introducción}

\cite{redmon2016yolo}

\section{Contexto y motivación}

\section{Estructura del documento}

% ==============================================================================
% Capítulo 2: Antecedentes
% ==============================================================================
\chapter{Antecedentes}

\section{Trabajos relacionados}

\section{Soluciones comerciales existentes}

\section{Estado del arte}

\section{Análisis comparativo}

% ==============================================================================
% Capítulo 3: Planteamiento del Problema
% ==============================================================================
\chapter{Planteamiento del Problema}

\section{Áreas protegidas y conservación de fauna silvestre}

\section{Problemática de la vigilancia en áreas remotas}

\section{Sistemas de monitoreo tradicionales}
\subsection{Cámaras trampa convencionales}
\subsection{Limitaciones actuales}

\section{Necesidad de detección de intrusos}

\section{Justificación del proyecto}

% ==============================================================================
% Capítulo 4: Objetivos y Alcance
% ==============================================================================
\chapter{Objetivos y Alcance}

\section{Objetivo general}

\section{Objetivos específicos}

\section{Alcance del proyecto}

\section{Limitaciones}

% ==============================================================================
% Capítulo 5: Marco Teórico
% ==============================================================================
\chapter{Marco Teórico}

\section{Internet de las Cosas (IoT)}
\subsection{Arquitecturas IoT}
\subsection{Protocolos de comunicación inalámbrica}

\section{Redes Mesh}
\subsection{Topologías de red}
\subsection{ESP-MESH y Mwifi}

\section{Inteligencia Artificial aplicada a visión por computadora}
\subsection{Redes neuronales convolucionales (CNN)}
\subsection{Detección de objetos con YOLO}
\subsection{SpeciesNet de Google}

\section{Tecnologías de desarrollo}
\subsection{Microcontroladores ESP32}
\subsection{ESP-IDF y ESP-MDF}
\subsection{Contenedorización con Docker}
\subsection{Framework Django}

% ==============================================================================
% Capítulo 6: Metodología
% ==============================================================================
\chapter{Metodología}

\section{Enfoque metodológico}

\section{Etapas del desarrollo}

\section{Herramientas y tecnologías utilizadas}

\section{Métricas de evaluación}

% ==============================================================================
% Capítulo 7: Diseño del Sistema
% ==============================================================================
\chapter{Diseño del Sistema}

\section{Arquitectura general}

\section{Diseño del hardware}
\subsection{Selección de componentes}
\subsection{Nodo de captura con cámara}
\subsection{Nodo raíz}
\subsection{Alimentación y consumo energético}

\section{Diseño de la red mesh}
\subsection{Topología de la red}
\subsection{Protocolo de comunicación}
\subsection{Formato de datos}

\section{Diseño del servicio de detección}
\subsection{Servidor de inferencia con SpeciesNet}
\subsection{Detección de animales, humanos y vehículos}
\subsection{Anotación de imágenes con bounding boxes}

\section{Diseño del servidor de aplicación}
\subsection{Arquitectura de servicios}
\subsection{Gestión de imágenes}
\subsection{Interfaz web}
\subsection{Bot de Telegram y sistema de alertas}

% ==============================================================================
% Capítulo 8: Implementación
% ==============================================================================
\chapter{Implementación}

\section{Nodo mesh (mesh-node)}
\subsection{Firmware del nodo de captura}
\subsection{Captura de imágenes}
\subsection{Compresión y transmisión}

\section{Nodo raíz (root-node)}
\subsection{Firmware del nodo raíz}
\subsection{Conexión con servidor TCP}
\subsection{Gestión de la red mesh}

\section{Servicio de detección (wildlife-detection)}
\subsection{Contenedor Docker con SpeciesNet}
\subsection{API de inferencia con LitServe}
\subsection{Procesamiento de imágenes}

\section{Servidor de aplicación (server)}
\subsection{Aplicación Django}
\subsection{Integración con SpeciesNet}
\subsection{Bot de Telegram y sistema de alertas}
\subsection{Base de datos PostgreSQL}
\subsection{Despliegue con Docker Compose}

% ==============================================================================
% Capítulo 9: Pruebas y Resultados
% ==============================================================================
\chapter{Pruebas y Resultados}

\section{Ambiente de pruebas}

\section{Pruebas de conectividad y red mesh}
\subsection{Alcance de la red}
\subsection{Latencia de transmisión}
\subsection{Estabilidad de la conexión}

\section{Pruebas de detección}
\subsection{Detección de fauna silvestre}
\subsection{Detección de humanos}
\subsection{Detección de vehículos}

\section{Evaluación del modelo de IA}
\subsection{Precisión y recall}
\subsection{Tiempo de inferencia}

\section{Pruebas de consumo energético}

\section{Pruebas del sistema de alertas}
\subsection{Tiempo de respuesta}

\section{Análisis de resultados}

% ==============================================================================
% Capítulo 10: Conclusiones
% ==============================================================================
\chapter{Conclusiones}

\section{Conclusiones generales}

\section{Aportes del trabajo}

\section{Trabajos futuros}

\section{Recomendaciones}

% ==============================================================================
% Referencias
% ==============================================================================
\printbibliography[heading=bibintoc]

% ==============================================================================
% Anexos
% ==============================================================================
\appendix

\chapter{Esquemáticos del hardware}

\chapter{Código fuente relevante}
\section{Firmware del nodo mesh}
\section{Firmware del nodo raíz}
\section{Servidor de detección}
\section{Aplicación Django}

\chapter{Manual de instalación y configuración}
\section{Configuración del firmware}
\section{Despliegue del servidor}
\section{Configuración del bot de Telegram}

\chapter{Manual de usuario}

\chapter{Especificaciones técnicas}

\chapter{Análisis de viabilidad económica}

\end{document}