% Tipo de documento y márgenes
\documentclass[a4paper, 12pt]{report}
\usepackage[top=3cm, bottom=3cm, left = 2cm, right = 2cm]{geometry}

% Codificación y lenguaje
\usepackage[utf8]{inputenc}
\usepackage{csquotes}
\usepackage[spanish]{babel}

% Imágenes
\usepackage{graphicx}
\usepackage{subcaption}

% Bibliografía
\usepackage[nottoc]{tocbibind}
\usepackage[style=ieee, backend=biber]{biblatex}
\addbibresource{referencias.bib}

% Links
\usepackage{hyperref}
\hypersetup{colorlinks=true,linkcolor=black,citecolor=black,urlcolor=blue}

% Glosario
\usepackage{datatool}[=v2.32]
\usepackage[toc,acronym]{glossaries}
\setacronymstyle{long-short}
\makenoidxglossaries
\loadglsentries{glosario}

\usepackage[locale=DE]{siunitx}

\title{Sistema de Monitoreo y Alerta Temprana basado en Inteligencia Artificial
para Áreas Protegidas}
\author{Autores:\\Fabrizio Martin Contigiani\\Gabriel Orlando Da
Silva Schmies \\\\Tutor:\\Dr. Ing. Sergio Eduardo Moya}
\date{\today}

\begin{document}

\maketitle

\tableofcontents

\begin{abstract}
	Esta primera versión del borrador se ha centrado primordialmente en
	la definición de la estructura general y la articulación del
	contenido central del documento.
	En consecuencia, se priorizaron estos aspectos, dejando la
	implementación de detalles de formato, tales como la carátula, el
	encabezado y el pie de página, para una próxima iteración o entrega.
	Asimismo, las secciones que aún requieren de ser escritas han sido
	completadas temporalmente con texto de relleno para poder
	visualizar correctamente la estructura del documento.

	\vspace{2em}
	\noindent\textbf{Palabras Clave} - Cámaras Trampa, Internet de las Cosas,
	Inteligencia Artificial, Monitoreo de Fauna, Detección de Intrusos,
	Vigilancia, Wi-Fi Mesh
\end{abstract}

\tableofcontents

\listoffigures

\listoftables

\printnoidxglossary[type=main]
\printnoidxglossary[type=\acronymtype]

\chapter{Introducción}

\gls{ia} \gls{miniz}

\end{document}